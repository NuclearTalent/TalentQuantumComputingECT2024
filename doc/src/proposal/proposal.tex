\documentclass[superscriptaddress,amsmath,amssymb,aps,floatfix]{revtex4-2}
\usepackage{graphicx,tikz} 
\usepackage[mode=buildmissing]{standalone} 
\usepackage{dcolumn}
\usepackage{bm}
\usepackage{physics}
\usepackage{siunitx}
\usepackage{soul}
\usepackage{comment}
\usepackage[bb=boondox]{mathalfa}
\usetikzlibrary{external}
\tikzexternalize[prefix=figures/]
\DeclareMathOperator{\sinc}{sinc}
\newcommand{\may}[1]{\textcolor{orange}{#1}}
\newcommand{\ale}[1]{{\color{blue}{\bf Ale}: #1}}
\newcommand{\zd}[1]{{\color{purple}{\bf ZD}: #1}}
\usepackage{xcolor}

\begin{document}
\title{Proposal for a Nuclear Talent course at the ECT* in 2025:\\Quantum Computing for Nuclear
Physics}


\author{Alexei Bazavov}

\affiliation{Department of Computational Mathematics, Science and
Engineering and Department of Physics and Astronomy, Michigan State
University, East Lansing, MI 48824, USA}

\author{Zohreh Davoudi}
\affiliation{Department of Physics, University of Maryland, College Park,
MD 20742, USA}
\affiliation{Center for Quantum Information and Computer Science (QuICS), University of Maryland, College Park, MD 20742, USA}

\author{Morten Hjorth-Jensen}
\affiliation{Department of Physics and Center for Computing in Science Education, University of Oslo, N-0316 Oslo, Norway}
\affiliation{Facility for Rare Isotope Beams and Department of Physics and Astronomy, Michigan State University, East Lansing, MI 48824, USA}

\author{Ryan LaRose}
\affiliation{Department of
Computational Mathematics, Science and Engineering, Michigan State
University, East Lansing, MI 48824, USA}

\author{Dean Lee}
\affiliation{Facility for Rare Isotope Beams and Department of Physics and
Astronomy, Michigan State University, East Lansing, MI 48824, USA}

\author{Alessandro Roggero} 
\affiliation{Department of Physics, University of Trento, Povo, 38123 Trento, Italy}
    
\maketitle

\section{Proposal for a Nuclear Talent course at the ECT* in 2025}

We would like to propose a three-week Nuclear Talent course on theory
for exploring quantum computing and quantum information science
applied to nuclear physics at the ECT* for the summer of 2025. We
believe such a course has a strong potential to attract many students,
theorists and experimentalists alike.

Below we give a short motivation for the proposed course and the
rationale behind the Nuclear Talent initiative. Thereafter we detail
our course plans with learning outcomes, objectives and teaching
philosophy, as well as various organizational and practical matters.

The teaching teams consists of quantum computing theorists and nuclear
theorists, with expertise ranging from many-body methods to quantum
field theories. Several of us have developed courses on quantum
computing and/or taught similar courses, in addition to our ongoing
research on quantum computing. This spans from quantum engineering to
developments of new algorithms and error correction studies. We
believe such a mix is important as it gives the students a better
understanding on how quantum computing can be applied to nuclear
physics problems, and what are the limitations and possibilities in
understanding and interpreting the various algorithms on present and
planned quantum technologies.

\subsection{Motivation}\label{motivation}

For nuclear theorists, the overarching challenge is to develop a
comprehensive description of nuclei and their reactions, grounded in
the fundamental interactions between the constituent nucleons with
quantifiable uncertainties to maximize predictive power. As
experimental frontiers have shifted to the study of rare isotopes, the
predictive power of successful phenomenological approaches like the
shell model and density functional theory is challenged by the
scarcity of nearby experimental data to constrain model
parameters. Therefore, it is expected that few-body and many-body
methods will play an increasingly prominent role to help improve the
predictive power of such ``data driven'' methods as experiment moves
deeper into largely unexplored regions of the nuclear chart.

To understand why nuclear matter is stable, and thereby shed light on
the limits of nuclear stability, is one of the overarching aims and
intellectual challenges of basic research in nuclear physics. To
relate the stability of matter to the underlying fundamental forces
and particles of nature as manifested in nuclear matter, is central to
present and planned rare isotope facilities. From a theoretical
standpoint, this involves understanding how the basic building blocks
of Nature interact and conspire to build up atomic nuclei as we know
them, with the aim to understand what makes visible matter stable. The
theoretical efforts span from methods like lattice quantum
chromodynamics, via effective field theories to many-body theories
applied to atomic nuclei and infinite nuclear matter. All these
methods rely on theoretical approximations whose applicabilities are
often limited by the dimensionality of the specific problem being
studied. In recent years, there has been considerable progress in
developing quantum-computing algorithms applied to quantum many-body
systems, with the hope to circumvent many of the classically
intractable problems.

This proposal for a Nuclear Talent school aims at bringing together
the efforts of nuclear many-body theorists, quantum information
theorists, and mathematicians in order to present and discuss
algorithms for studying nuclear systems using recent progress in
quantum information theory.

\subsection{Introduction to the Talent Courses}

The TALENT initiative, \href{http://www.nucleartalent.org}{Training in
  Advanced Low Energy Nuclear Theory}, aims at providing an advanced
and comprehensive training to graduate students and young researchers
in low-energy nuclear theory. The initiative is a multinational
network of several European and Northern American institutions and
aims at developing a broad curriculum that will provide the necessary
training in cutting-edge theory for understanding nuclei and nuclear
reactions.  These objectives will be met by offering a series of
lectures, delivered by experts in nuclear many-body theory and quantum
information theories.  The educational material generated under this
program will be collected in the form of WEB-based courses, textbooks,
and a variety of modern educational resources. No such
all-encompassing material is available at present; its development
will allow dispersed university groups to profit from the best
expertise available worldwide.

The Nuclear Talent initiative has organized and run several advanced
courses since the summer of 2012. Several of these courses have been run
and organized (in a very successful way) at the premises of the
ECT\emph{. We hope thus, if this course gets approved by the board of
directors of the ECT}, that we can continue this successful and very
fruitful collaboration.

\section{Aims and Learning Outcomes}
\subsection{Format}

We propose approximately forty-five hours of lectures over three weeks
and a comparable amount of practical computer and exercise sessions with
supervised practices and tutorials.

The mornings will consist of lectures and the afternoons will be devoted
to exercises meant to shed light on the exposed theory, the
computational projects, and individual student projects. These components
will be coordinated to foster student engagement, maximize learning, and
create lasting value for the students. For the benefit of the TALENT
series and of the community, material (courses, slides, problems and
solutions, reports on students' projects) will be made publicly
available using version control software like \emph{git} and posted
electronically on \href{https://github.com}{github}.

As with previous TALENT courses, we envision the following features for
the afternoon sessions: We will use both individual and group work to
carry out tasks that are very specific in technical instructions, but
leave freedom for creativity.

\begin{itemize}
\item
  Groups will be carefully put together to maximize diversity of
  backgrounds.
\item
  Results will be presented in a conference-like setting to create
  accountability.
\item
  We will organize events where individuals and groups exchange their
  experiences, difficulties, and successes to foster interaction.
\item
  During the school, on-line and lecture-based training tailored to
  technical issues will be provided. Students will learn to use and
  interpret the results of computer-based and hands-on calculations of
  quantum computing algorithms. The lectures will be aligned with the
  practical computational projects and exercises and the lecturers will
  be available to help students and work with them during the exercise
  sessions.
\item
  These interactions will raise topics not originally envisioned for the
  course but which are recognized to be valuable for the students. There
  will be flexibility to organize mini-lectures and discussion sessions
  on an ad-hoc basis in such cases.

\item
  Training modules, codes, lectures, practical exercise instructions,
  online logbooks, instructions and information created by participants
  will be merged into a comprehensive website that will be available to
  the community and the public for self-guided training or for use in
  various educational settings (for example, a graduate course at a
  university could assign some of the projects as homework or an extra
  credit project, etc).
\end{itemize}


\subsection{Course content, learning outcomes and detailed plan}

The course will be taught as an intensive course of duration of three
weeks, with a total time of 45 h of lectures, 45 h of exercises and a
final assignment of 2 weeks of work. The total load will be
approximately 160-170 hours, corresponding to \textbf{7 ECTS} in
Europe.  The final assignment will be graded with marks A, B, C, D, E
and failed for Master students and passed/not passed for PhD
students. A course certificate will be issued for students requiring
it from the University of Trento.

The organization of a typical course day is as follows:

\begin{enumerate}
\def\labelenumi{\arabic{enumi}.}
\item
  9am-12pm: Lectures, project relevant information and directed
  exercises
\item
  12pm-2pm: Lunch
\item
  2pm-6pm: Computational projects, exercises and hands-on sessions
\item
  6pm-7pm: Wrap-up of the day and eventual student presentations
\end{enumerate}

If approved by the ECT* board of directors, our preferred time slot
would be from the second half of June till the second half of July.


\subsubsection{First week, schedule and learning outcomes}

\begin{table}[hbtp]
\begin{tabular}{|l|l|l|l|l|} \hline
    & First session  & Second session  & Exercises and project work & Student presentations \\ \hline
  Monday & Review of Linear Algebra and & Qubits, measurements and  & Work on codes for basis & \\
         & density matrices and states & quantum gates and circuits & and quantum gates & \\
  Tuesday & Quantum Fourier & Quantum phase  & QFT exercises &  \\
          & transforms (QFT) & estimation algorithm (QPE) & and codes & \\
  Wednesday  & Quantum algorithms &  Quantum advantage & Work on exercises & \\    
  Thursday & Variational Quantum & Simple Hamiltonians & Implementing QPE & \\
            & Eigensolver (VQE) & & Work  on VQE & \\
  Friday  & Nuclear physics  & Lipkin model & VQE implementation & \\
           & Hamiltonians     & and Rodeo algorithm  & of the Lipkin model & \\ \hline
\end{tabular}
\caption{Teaching schedule first week}
\end{table}

The first week focuses, after a reminder of central linear algebra
elements, on basic ingredients of quantum computing such as rewriting
quantum mechanical operations as quantum gates and circuits, how to
perform measurements and how to obtain eigenvalues of selected
Hamiltonians. We will also (Wednesday) discuss some selected quantum
algorithms, such as Grover's algorithm, Simon's algorithms and quantum
advantage via Shor's algorithm.


To obtain the eigenvalues we will discuss the quantum
phase estimation algorithm (which requires a discussion of Quantum
Fourier transforms) and the widely used variational quantum
eigensolver (VQE). After having introduced some simpler Hamiltonians
defined by various Pauli matrices, we will demonstrate how to rewrite
a widely used Hamiltonian given by a second-quantized representation
in terms of various Pauli matrices. The Hamiltonian we will focus on
the first week is the so-called Lipkin Hamiltonian which does not require
a so-called Jordan-Wigner transformation. The latter transformation will
be discussed during the second week.  The students will work on
analytical exercises as well as computational exercises. The latter
will focus on developing a code which implements the VQE method for
finding the eigenvalues of the above Hamiltonians.  The Rodeo algorithm will also be discussed. This code can be
extended upon and can be used to define a final project students can
hand in for final credits.

Many of the topics discussed during the first week, will serve as
background material for the next two weeks.

The schedule for the student presentations will be finalized during
the Talent course.

\subsubsection{Second week, schedule and learning outcomes}


\begin{table}[hbtp]
\begin{tabular}{|l|l|l|l|l|} \hline
& First session  & Second session  & Exercises and project work & Student presentations \\ \hline
  Monday  & Hamiltonian dynamics & Lie-Trotter-Suzuki & Work on VQE& \\
          &                      & formulas for simulations  & and exercises & \\ 
  Tuesday & Encoding (relativistic and nonrelativistic)  & Encoding & Exercises & \\
          & fermions and bosons on quantum computers     & second part & and project work & \\              
  Wednesday & Quantum simulation & Quantum simulation  & Simulations  & \\
            & of pionless EFT    & of pionless EFT, part 2 & of pionless EFT & \\
  Thursday & Nuclear response & Neutrino dynamics  & Exercises and & \\
           & functions  &  in dense environments & project work & \\
  Friday & Noise mitigation and & Quantum error correction   & Exercises and project work & \\ 
  & NISQ computing & fault tolerance  &  & \\ \hline  
\end{tabular}
\caption{Tentative schedule second  week}
\end{table}

The second week starts with a discussion of product formulae such as
the Lie-Trotter-Suzuki approximation and how to simulate Hamiltonian
dynamics.  Thereafter we discuss in detail how to encode fermionic and
bosonic systems through for example the so-called Jordan-Wigner
transformation. This will allow us to study more general Hamiltonians
such as a pionless EFT based Hamiltonian and nuclear response function
and neutrino dynamics.  Entanglement in nuclear many-body systems will
also be discussed before we wrap up the week with a discussion of
noise mitigation and quantum error correction algorithms.

Given the more general Hamiltonians, the codes developed during the
first week can be extended to include more Hamiltonians and systems of
relevance for nuclear physics.


\subsubsection{Third week, schedule and learning outcomes}


\begin{table}[hbtp]
\begin{tabular}{|l|l|l|l|l|} \hline
    & First session  & Second session  & Exercises and project work & Student presentations \\ \hline
  Monday & Quantum simulation of   & Quantum simulation of   & Exercises & \\
         & scattering in scalar field theory, I & scattering in scalar field theory, II & and project work & \\
  Tuesday & Hamiltonian formulations of & Hamiltonian formulations of & Exercises & \\
          & gauge theories (Abelian)    & gauge theories (non-Abelian) & and project work & \\
  Wednesday & Time evolution  & Time evolution & Exercises  & \\
            & in gauge theoris & in gauge theories & and project work \\
  Thursday & Applications to QFT & Applications to QFT& Project work & \\
  Friday & Summary of course & Discussion of projects  & Project work & \\ \hline
\end{tabular}
\caption{Teaching schedule third  week}
\end{table}

Depending on our progress during the first two weeks, the schedule of
the last week may be subject to changes. The tentative plan for the
final week is dedicate it to a discussion of quantum computing for
quantum (gauge) field theories. Here the learning outcomes will focus
on quantum simulations of scattering in scalar field theory,
Hamiltonian formulations of gauge theories and time evolution in gauge
theories, Abelian and non-Abelian, with final applications. The course
ends with a summary and discussions of the projects.




\subsection{Instructors and organizers}\label{instructors-and-organizers}

The organizers and instructors are


\begin{enumerate}
\item 
\href{https://directory.natsci.msu.edu/Directory/Profiles/Person/101033}{Alexei
Bazavov} at Department of Computational Mathematics, Science and
Engineering and Department of Physics and Astronomy, Michigan State
University, East Lansing, MI 48824, USA. He is a theoretical particle physicist specializing in study of strongly coupled theories, in particular, Quantum Chromodynamics. Items of particular interest to him include quantum computing and algorithms, quantum field theory, finite-temperature field theory, lattice gauge theory with applications to particle and nuclear physics, parallel algorithms, iterative solvers, molecular dynamics algorithms, inverse problems and Bayesian inference, ultra-cold atomic systems and quantum simulations and effective field theory.

\item
  \href{https://umdphysics.umd.edu/people/faculty/current/item/927-davoudi.html}{Zohreh
  Davoudi}, Department of Physics and Center for Quantum Information and Computer Science (QuICS), University of Maryland, College Park,
  MD 20742, USA. Davoudi is an expert in lattice QCD for nuclear physics. She further
  specializes in quantum simulation and computing for quantum field
  theories, including lattice gauge theories and effective field
  theories of nuclear physics.

\item
  \href{http://mhjgit.github.io/info/doc/web/}{Morten Hjorth-Jensen} at
  Facility for Rare Isotope Beams and Department of Physics and
  Astronomy, Michigan State University, East Lansing, MI 48824, USA \&
  Department of Physics, University of Oslo, N-0316 Oslo, Norway. Hjorth-Jensen has his background in studies of different many-body theories applied to problems in nuclear physics and condensed matter physics. He works also on quantum engineering and machine learning applied to many-body systems. He has over many years developed introductory and advanced learning material in many-body physics, quantum computing, computational physics and machine learning.
\item
  \href{https://frib.msu.edu/for-students/faculty/lee-profile}{Dean Lee}
  at Facility for Rare Isotope Beams and Department of Physics and
  Astronomy, Michigan State University, East Lansing, MI 48824, USA. His research is focused on connecting fundamental physics to forefront experiments. He studies many aspects of quantum few- and many-body systems. Together with collaborators, he has developed lattice Monte Carlo methods that probe strongly-interacting systems and study superfluidity, nuclear clustering, phase transitions, and other emergent phenomena from first principles. He  is also engaged in novel applications of new technologies for scientific research. This includes new algorithms for quantum computing and the development of emulators and machine learning algorithms based on concepts such as eigenvector continuation.
\item
  \href{https://www.ryanlarose.com/}{Ryan LaRose} at at Department of
  Computational Mathematics, Science and Engineering, Michigan State
  University, East Lansing, MI 48824, USA. He does research in computational physics and quantum information science. He is  interested in both the physics of computation and the computation of physics - that is, what quantum physics can tell us about information and computer science, and how quantum computers can solve practical problems in physics and related fields.
\item
  \href{https://webapps.unitn.it/du/en/Persona/PER0016084/Didattica}{Alessandro
  Roggero}, Department of Physics, University of Trento, Povo, 38123
  Trento, Italy. He works on  simulations of strongly correlated quantum many-body systems using a combination of classical techniques and calculations on
quantum devices. In particular, in understanding the role of entanglement in both observable properties of physical systems and as a fundamental tool to understand the structure of many-body states. His research interests include quantum simulations of inelastic nuclear processes, neutrinos in dense matter, quantum computing and algorithms, quantum Monte Carlo methods and effective field theory.

\item
  An eventual Post-doctoral fellow and an advanced graduate student as
  teaching assistants.
\end{enumerate}

Morten Hjorth-Jensen Alessandro Roggero will also function as student
advisors and coordinators.

\subsection{Audience and Prerequisites}

Students and post-doctoral fellows interested in quantum computing
applied to nuclear physics problems, from nuclear structure to quantum
field theories. The material will be of interest, and accessible, to
both theorists and experimentalists, and will include learning the
practical use of quantum computing software in order to interpret and
study nuclear systems.

The students are expected to have operating programming skills in
compiled programming languages like Fortran or C++ or preferentially in
an interpreted language like Python and knowledge of quantum mechanics
at an intermediate level.

\subsection{Admission}

The target group is Master of Science students, PhD students and early
post-doctoral fellows. Also senior staff can attend but they have to be
self-supported. The maximum number of students is 20-30, of which
hopefully at most 15-20 can receive full local support.

The process of selections of the students will be managed in agreement
with the ECT*.

\subsection{Preliminary budget}

We expect to accept between 20-30 students. Local students from the
University of Trento are fully self-supported. If approved, we would
very much appreciate if the ECT* can sponsor 15-20 of the selected
students with local expenses, that is lodging and meals during weekdays.
Any additional funds for sponsoring further students is highly
appreciated.

All travel expenses will be covered by the respective home institute.
Instructors are self-supported. We plan to raise additional funds to
cover local support for additional students and the expenses of the
instructors.

There is no participation fee. Administrative support from the ECT* in
organizing the course and setting up the application procedure is
essential for a smooth (as always) outcome. The administrative
experience of the staff at the ECT* has been unique and essential in
running successfully our previous Talent courses. We would thus highly
appreciate it if these services are provided if the proposal is
approved.



\end{document}





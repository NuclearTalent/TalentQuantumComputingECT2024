
% LaTeX Beamer file automatically generated from DocOnce
% https://github.com/doconce/doconce

%-------------------- begin beamer-specific preamble ----------------------

\documentclass{beamer}

\usetheme{red_plain}
\usecolortheme{default}

% turn off the almost invisible, yet disturbing, navigation symbols:
\setbeamertemplate{navigation symbols}{}

% Examples on customization:
%\usecolortheme[named=RawSienna]{structure}
%\usetheme[height=7mm]{Rochester}
%\setbeamerfont{frametitle}{family=\rmfamily,shape=\itshape}
%\setbeamertemplate{items}[ball]
%\setbeamertemplate{blocks}[rounded][shadow=true]
%\useoutertheme{infolines}
%
%\usefonttheme{}
%\useinntertheme{}
%
%\setbeameroption{show notes}
%\setbeameroption{show notes on second screen=right}

% fine for B/W printing:
%\usecolortheme{seahorse}

\usepackage{pgf}
\usepackage{graphicx}
\usepackage{epsfig}
\usepackage{relsize}

\usepackage{fancybox}  % make sure fancybox is loaded before fancyvrb

\usepackage{fancyvrb}
%\usepackage{minted} % requires pygments and latex -shell-escape filename
%\usepackage{anslistings}
%\usepackage{listingsutf8}

\usepackage{amsmath,amssymb,bm}
%\usepackage[latin1]{inputenc}
\usepackage[T1]{fontenc}
\usepackage[utf8]{inputenc}
\usepackage{colortbl}
\usepackage[english]{babel}
\usepackage{tikz}
\usepackage{framed}
% Use some nice templates
\beamertemplatetransparentcovereddynamic

% --- begin table of contents based on sections ---
% Delete this, if you do not want the table of contents to pop up at
% the beginning of each section:
% (Only section headings can enter the table of contents in Beamer
% slides generated from DocOnce source, while subsections are used
% for the title in ordinary slides.)
\AtBeginSection[]
{
  \begin{frame}<beamer>[plain]
  \frametitle{}
  %\frametitle{Outline}
  \tableofcontents[currentsection]
  \end{frame}
}
% --- end table of contents based on sections ---

% If you wish to uncover everything in a step-wise fashion, uncomment
% the following command:

%\beamerdefaultoverlayspecification{<+->}

\newcommand{\shortinlinecomment}[3]{\note{\textbf{#1}: #2}}
\newcommand{\longinlinecomment}[3]{\shortinlinecomment{#1}{#2}{#3}}

\definecolor{linkcolor}{rgb}{0,0,0.4}
\hypersetup{
    colorlinks=true,
    linkcolor=linkcolor,
    urlcolor=linkcolor,
    pdfmenubar=true,
    pdftoolbar=true,
    bookmarksdepth=3
    }
\setlength{\parskip}{0pt}  % {1em}

\newenvironment{doconceexercise}{}{}
\newcounter{doconceexercisecounter}
\newenvironment{doconce:movie}{}{}
\newcounter{doconce:movie:counter}

\newcommand{\subex}[1]{\noindent\textbf{#1}}  % for subexercises: a), b), etc

%-------------------- end beamer-specific preamble ----------------------

% Add user's preamble




% insert custom LaTeX commands...

\raggedbottom
\makeindex

%-------------------- end preamble ----------------------

\begin{document}

% matching end for #ifdef PREAMBLE

\newcommand{\exercisesection}[1]{\subsection*{#1}}



% ------------------- main content ----------------------



% ----------------- title -------------------------

\title{Exercises June 16}

% ----------------- author(s) -------------------------

\author{Nuclear TALENT course on quantum computing\inst{}}
\institute{}
% ----------------- end author(s) -------------------------

\date{June 16, 2025
% <optional titlepage figure>
% <optional copyright>
}

\begin{frame}[plain,fragile]
\titlepage
\end{frame}

\begin{frame}[plain,fragile]
\frametitle{First exercise set}

The exercises we present each week are meant to build the basis for
the possible  project we will work. See in particular the third lecture today.
\end{frame}

\begin{frame}[plain,fragile]
\frametitle{Ex1: One-qubit basis and  Pauli matrices}

Write a function which sets up a one-qubit basis and apply the various Pauli matrices to these basis states.
\end{frame}

\begin{frame}[plain,fragile]
\frametitle{Ex2: Hadamard and Phase gates}

Apply the Hadamard and Phase gates to the same one-qubit basis states and study their actions on these states.
\end{frame}

\begin{frame}[plain,fragile]
\frametitle{Ex3: Traces of operators}

Prove that the trace is cyclic, that is for three operators $\bm{A}$, $\bm{B}$ and $\bm{C}$, we have
\[
\mathrm{Tr}\{\bm{ABC}\}=\mathrm{Tr}\{\bm{CAB}\}=\mathrm{Tr}\{\bm{BCA}\}.
\]
\end{frame}

\begin{frame}[plain,fragile]
\frametitle{Ex4: Exponentiated operators}

Let $\bm{A}$ be an operator on a vector space satisfying $\bm{A}^2=1$ and $\alpha$ any real constant. Show that
\[
\exp{\imath\alpha \bm{A}}=\sum_{n=0}^{\infty} \frac{(i\alpha)^n}{n!}\bm{A}^n=\bm{I}\cos{\alpha}+\imath\bm{A}\sin{\alpha}.
\]
Does this apply to the Pauli matrices?
\end{frame}

\begin{frame}[plain,fragile]
\frametitle{Ex5: Hamiltonians rewritten in terms of simple Pauli matrices}

We consider a simple $2\times 2$ real
Hamiltonian consisting of a diagonal part $H_0$ and off-diagonal part
$H_I$, playing the roles of a non-interactive one-body and interactive
two-body part respectively. Defined through their matrix elements, we
express them in the Pauli basis $\vert 0\rangle$ and $\vert 1 \rangle$

\begin{align*}
    \begin{split} 
        H &= H_0 + H_I \\
        H_0 = \begin{bmatrix}
            E_1 & 0 \\
            0 & E_2
        \end{bmatrix}&, \hspace{20px}
        H_I = \lambda \begin{bmatrix}
            V_{11} & V_{12} \\
            V_{21} & V_{22}
        \end{bmatrix}
    \end{split}
\end{align*}
Where $\lambda \in [0,1]$ is a coupling constant parameterizing the strength of the interaction.
\end{frame}

\begin{frame}[plain,fragile]
\frametitle{Rewriting in terms of Pauli matrices}

Define
\[
    E_{+} = \frac{E_1 + E_2}{2},\hspace{20px} E_{-} = \frac{E_1 - E_2}{2}
\]
show  that by combining the identity and $Z$ Pauli matrix, this can be expressed as

\[
    H_0 = E_{+} I + E_{-} Z
\]
\end{frame}

\begin{frame}[plain,fragile]
\frametitle{The interaction part}

For $H_1$ we use the same trick to fill the diagonal, defining $V_{+} = (V_{11} + V_{22})/2, V_{-} = (V_{11} - V_{22})/2$. From the hermiticity requirements of $H$, we note that $V_{12} = V_{21} \equiv V_o$. Use this to simplify the problem to a simple $X$ term. 

\[
    H_I = V_{+} I + V_{-} Z + V_o X
\]
\end{frame}

\begin{frame}[plain,fragile]
\frametitle{Measurement basis}

For the above system show that the Pauli $X$ matrix can be rewritten in terms of the Hadamard matrices and the Pauli $Z$ matrix, that is
\[
X=HZH.
\]
\end{frame}

\end{document}


% LaTeX Beamer file automatically generated from DocOnce
% https://github.com/doconce/doconce

%-------------------- begin beamer-specific preamble ----------------------

\documentclass{beamer}

\usetheme{red_plain}
\usecolortheme{default}

% turn off the almost invisible, yet disturbing, navigation symbols:
\setbeamertemplate{navigation symbols}{}

% Examples on customization:
%\usecolortheme[named=RawSienna]{structure}
%\usetheme[height=7mm]{Rochester}
%\setbeamerfont{frametitle}{family=\rmfamily,shape=\itshape}
%\setbeamertemplate{items}[ball]
%\setbeamertemplate{blocks}[rounded][shadow=true]
%\useoutertheme{infolines}
%
%\usefonttheme{}
%\useinntertheme{}
%
%\setbeameroption{show notes}
%\setbeameroption{show notes on second screen=right}

% fine for B/W printing:
%\usecolortheme{seahorse}

\usepackage{pgf}
\usepackage{graphicx}
\usepackage{epsfig}
\usepackage{relsize}

\usepackage{fancybox}  % make sure fancybox is loaded before fancyvrb

\usepackage{fancyvrb}
\usepackage{minted} % requires pygments and latex -shell-escape filename
%\usepackage{anslistings}
%\usepackage{listingsutf8}

\usepackage{amsmath,amssymb,bm}
%\usepackage[latin1]{inputenc}
\usepackage[T1]{fontenc}
\usepackage[utf8]{inputenc}
\usepackage{colortbl}
\usepackage[english]{babel}
\usepackage{tikz}
\usepackage{framed}
% Use some nice templates
\beamertemplatetransparentcovereddynamic

% --- begin table of contents based on sections ---
% Delete this, if you do not want the table of contents to pop up at
% the beginning of each section:
% (Only section headings can enter the table of contents in Beamer
% slides generated from DocOnce source, while subsections are used
% for the title in ordinary slides.)
\AtBeginSection[]
{
  \begin{frame}<beamer>[plain]
  \frametitle{}
  %\frametitle{Outline}
  \tableofcontents[currentsection]
  \end{frame}
}
% --- end table of contents based on sections ---

% If you wish to uncover everything in a step-wise fashion, uncomment
% the following command:

%\beamerdefaultoverlayspecification{<+->}

\newcommand{\shortinlinecomment}[3]{\note{\textbf{#1}: #2}}
\newcommand{\longinlinecomment}[3]{\shortinlinecomment{#1}{#2}{#3}}

\definecolor{linkcolor}{rgb}{0,0,0.4}
\hypersetup{
    colorlinks=true,
    linkcolor=linkcolor,
    urlcolor=linkcolor,
    pdfmenubar=true,
    pdftoolbar=true,
    bookmarksdepth=3
    }
\setlength{\parskip}{0pt}  % {1em}

\newenvironment{doconceexercise}{}{}
\newcounter{doconceexercisecounter}
\newenvironment{doconce:movie}{}{}
\newcounter{doconce:movie:counter}

\newcommand{\subex}[1]{\noindent\textbf{#1}}  % for subexercises: a), b), etc

%-------------------- end beamer-specific preamble ----------------------

% Add user's preamble




% insert custom LaTeX commands...

\raggedbottom
\makeindex

%-------------------- end preamble ----------------------

\begin{document}

% matching end for #ifdef PREAMBLE

\newcommand{\exercisesection}[1]{\subsection*{#1}}



% ------------------- main content ----------------------



% ----------------- title -------------------------

\title{Quantum gates and quantum circuits and applications, summary from Monday}

% ----------------- author(s) -------------------------

\author{Nuclear TALENT course on quantum computing\inst{}}
\institute{}
% ----------------- end author(s) -------------------------

\date{Tuesday June 17, 2025
% <optional titlepage figure>
% <optional copyright>
}

\begin{frame}[plain,fragile]
\titlepage
\end{frame}

\begin{frame}[plain,fragile]
\frametitle{Quantum Control}

The three basic steps: 
\begin{enumerate}
\item Initialization (see for example \href{{https://journals.aps.org/prd/pdf/10.1103/PhysRevD.111.074515}}{\nolinkurl{https://journals.aps.org/prd/pdf/10.1103/PhysRevD.111.074515}})

\item Evolution of the system

\item Measurement/readout
\end{enumerate}

\noindent
In this lecture we will illustrate this using some simple examples
tailored to \textbf{quantum sensing}. We will also remind you of how we
rephrase the physical processes in terms og gates and circuits.

Let us start with a reminder from yesterday about various gates and how to encode them.
\end{frame}

\begin{frame}[plain,fragile]
\frametitle{Widely used gates}

There are several widely used quantum gates. Perhaps the most famous are 
the Pauli gates correspond to the Pauli matrices

\[
I=\begin{bmatrix} 1 & 0 \\ 0 & 1 \end{bmatrix},
\]

\[
X =\begin{bmatrix} 0 & 1 \\ 1 & 0\end{bmatrix},
\]

\[
Y=\begin{bmatrix}0 & -i \\i & 0\end{bmatrix},
\]

\[
Z=\begin{bmatrix} 1 & 0 \\ 0 & -1\end{bmatrix}.
\]
\end{frame}

\begin{frame}[plain,fragile]
\frametitle{Algebra basis}

These gates form a basis for
the algebra $\mathfrak{su}(2)$. Exponentiating them will thus give us
a basis for SU(2), the group within which all single-qubit gates
live.
\end{frame}

\begin{frame}[plain,fragile]
\frametitle{Exponentiated Pauli gates}

These exponentiated Pauli gates are called rotation gates
$R_{\sigma}(\theta)$ because they rotate the quantum state around the
axis $\sigma=X,Y,Z$ of the Bloch sphere by an angle $\theta$. They are
defined as

\[
R_X(\theta)=e^{-i\frac{\theta}{2}X}=
\begin{bmatrix}
\cos\frac{\theta}{2} & -i\sin\frac{\theta}{2} \\
-i\sin\frac{\theta}{2} & \cos\frac{\theta}{2} 
\end{bmatrix},
\]
\[
R_Y(\theta)=e^{-i\frac{\theta}{2}Y}=
\begin{bmatrix}
\cos\frac{\theta}{2} & -\sin\frac{\theta}{2} \\
\sin\frac{\theta}{2} & \cos\frac{\theta}{2} 
\end{bmatrix},
\]
\[
R_Z(\theta)=e^{-i\frac{\theta}{2}Z}=\begin{bmatrix}
e^{-i\theta/2} & 0 \\
0 & e^{i\theta/2}\end{bmatrix}.
\]
\end{frame}

\begin{frame}[plain,fragile]
\frametitle{Basis for $\mathrm{SU}(2)$}

Because they form a basis for $\mathrm{SU}(2)$, any single-qubit gate
can be decomposed into three rotation gates. Indeed
\[
R_z(\phi)R_y(\theta)R_z(\lambda)=
\begin{bmatrix}
e^{-i\phi/2} & 0 \\
0 & e^{i\phi/2}
\end{bmatrix}
\begin{bmatrix}
\cos\frac{\theta}{2} & -\sin\frac{\theta}{2} \\
\sin\frac{\theta}{2} & \cos\frac{\theta}{2} 
\end{bmatrix}
\begin{bmatrix}
e^{-i\lambda/2} & 0 \\
0 & e^{i\lambda/2}
\end{bmatrix}
\]
which we can rewite as
\[
e^{-i(\phi+\lambda)/2}
\begin{bmatrix}
\cos\frac{\theta}{2} & -e^{i\lambda}\sin\frac{\theta}{2}\\
e^{i\phi}\sin\frac{\theta}{2} & e^{i(\phi+\lambda)}\cos\frac{\theta}{2}
\end{bmatrix},
\]

which is, up to a global phase, equal to the expression for an arbitrary single-qubit gate.
\end{frame}

\begin{frame}[plain,fragile]
\frametitle{Two-Qubit Gates}

A two-qubit gate is a physical action that is applied to two
qubits. It can be represented by a matrix $U$ from the group
SU(4). One important type of two-qubit gates are controlled gates,
which work as follows: Suppose $U$ is a single-qubit gate. A
controlled-$U$ gate ($CU$) acts on two qubits: a control qubit
$\vert x \rangle $ and a target qubit $\vert y \rangle $. The controlled-$U$ gate
applies the identity $I$ or the single-qubit gate $U$ to the target
qubit if the control gate is in the zero state $\vert 0\rangle$ or the one
state $\vert 1\rangle$, respectively.
\end{frame}

\begin{frame}[plain,fragile]
\frametitle{Control qubit}

The control qubit is not acted
upon. This can be represented as follows if
\[CU\vert xy\rangle=
\vert xy\rangle \hspace{0.1cm} \mathrm{if} \hspace{0.1cm}  \vert x \rangle =\vert 0\rangle.
\]
\end{frame}

\begin{frame}[plain,fragile]
\frametitle{In matrix form}

It is easier to see in a matrix form.
It can be written in matrix form by writing it as a superposition of
the two possible cases, each written as a simple tensor product

\[
CU = \vert 0\rangle\langle 0\vert\otimes I + \vert 1\rangle\langle 1 \vert \otimes U=\begin{bmatrix}
1 & 0 & 0 & 0 \\
0 & 1 & 0 & 0 \\
0 & 0 & u_{00} & u_{01} \\
0 & 0 & u_{10} & u_{11}
\end{bmatrix}.
\]
\end{frame}

\begin{frame}[plain,fragile]
\frametitle{CNOT gate}

One of the most fundamental controlled gates is the CNOT gate. It is
defined as the controlled-$X$ gate $CX$. It can be written in matrix form as follows:

\[
\mathrm{CNOT}=\mathrm{CX}=\begin{bmatrix}
1 & 0 & 0 & 0 \\
0 & 1 & 0 & 0 \\
0 & 0 & 0 & 1 \\
0 & 0 & 1 & 0
\end{bmatrix}.
\]
\end{frame}

\begin{frame}[plain,fragile]
\frametitle{$\mathrm{CX}$ gate}

It changes, when operating on a two-qubit state where the first qubit is the control qubit and the second qubit is the target qubit, the states (check this)
\[
\mathrm{CX}\vert 00\rangle=\vert 00\rangle,
\]
\[
\mathrm{CX}\vert 10\rangle= \vert 11\rangle,
\]
\[
\mathrm{CX}\vert 01\rangle= \vert 01\rangle,
\]
\[
\mathrm{CX}\vert 11\rangle= \vert 10\rangle,
\]
which you can easily see by simply multiplying the above matrix with any of the above states.
\end{frame}

\begin{frame}[plain,fragile]
\frametitle{Swap gate}

A widely used two-qubit gate that goes beyond the simple controlled function is the SWAP gate. It swaps the states of the two qubits it acts upon

\[
\mathrm{SWAP}\vert xy\rangle=\vert yx\rangle.
\]
and has the following matrix form

\[
\mathrm{SWAP}
=\begin{bmatrix}
1 & 0 & 0 & 0 \\
0 & 0 & 1 & 0 \\
0 & 1 & 0 & 0 \\
0 & 0 & 0 & 1
\end{bmatrix}.
\]
\end{frame}

\begin{frame}[plain,fragile]
\frametitle{An example of an OO  code for quantum gates and circuits}

\begin{minted}[fontsize=\fontsize{9pt}{9pt},linenos=false,mathescape,baselinestretch=1.0,fontfamily=tt,xleftmargin=2mm]{python}

import numpy as np
import random
import matplotlib.pyplot as plt
from collections import Counter
from mpl_toolkits.mplot3d import Axes3D # Ensure this is imported

# =============================== #
#       Quantum Gate Classes      #
# =============================== #

class Gate:
    def __init__(self, matrix, targets):
        self.matrix = np.array(matrix, dtype=np.complex128)
        # Convert Qubit objects (if any) to indices
        self.targets = [(t.index if isinstance(t, Qubit) else t) for t in targets]
        self.num_targets = len(self.targets)
        self.name = "CustomGate"

class OneQubitGate(Gate):
    def __init__(self, matrix, target):
        super().__init__(matrix, [target])
        # Add name attribute in subclasses if needed for representation
        if np.array_equal(matrix, np.eye(2)): self.name = "I"
        elif np.array_equal(matrix, np.array([[0,1],[1,0]])): self.name = "X"
        elif np.array_equal(matrix, np.array([[0,-1j],[1j,0]])): self.name = "Y"
        elif np.array_equal(matrix, np.array([[1,0],[0,-1]])): self.name = "Z"
        elif np.allclose(matrix, (1/np.sqrt(2))*np.array([[1,1],[1,-1]])): self.name = "H"
        elif np.array_equal(matrix, np.array([[1,0],[0,1j]])): self.name = "S"
        elif np.allclose(matrix, np.array([[1,0],[0,np.exp(1j*np.pi/4)]])): self.name = "T"
        # Note: Naming Rx, Ry, Rz requires theta, which is not stored in the base matrix

class TwoQubitGate(Gate):
    def __init__(self, matrix, control, target):
        super().__init__(matrix, [control, target])
        # Add name attribute in subclasses if needed for representation
        if np.array_equal(matrix, np.array([[1,0,0,0],[0,1,0,0],[0,0,0,1],[0,0,1,0]])): self.name = "CNOT"
        elif np.array_equal(matrix, np.array([[1,0,0,0],[0,1,0,0],[0,0,1,0],[0,0,0,-1]])): self.name = "CZ"
        elif np.array_equal(matrix, np.array([[1,0,0,0],[0,0,1,0],[0,1,0,0],[0,0,0,1]])): self.name = "SWAP"


# One-qubit standard gate matrices
def I():  return np.eye(2)
def X():  return np.array([[0,1],[1,0]])
def Y():  return np.array([[0,-1j],[1j,0]])
def Z():  return np.array([[1,0],[0,-1]])
def H():  return (1/np.sqrt(2))*np.array([[1,1],[1,-1]])
def S():  return np.array([[1,0],[0,1j]])
def T():  return np.array([[1,0],[0,np.exp(1j*np.pi/4)]])

def Rx(theta):
    return np.array([
        [np.cos(theta/2), -1j*np.sin(theta/2)],
        [-1j*np.sin(theta/2), np.cos(theta/2)]
    ])

def Ry(theta):
    return np.array([
        [np.cos(theta/2), -np.sin(theta/2)],
        [np.sin(theta/2),  np.cos(theta/2)]
    ])

def Rz(theta):
    return np.array([
        [np.exp(-1j*theta/2), 0],
        [0, np.exp(1j*theta/2)]
    ])

# Two-qubit gate matrices
def CNOT_matrix(): # Renamed to avoid conflict if CNOTGate is a class
    return np.array([
        [1,0,0,0],
        [0,1,0,0],
        [0,0,0,1],
        [0,0,1,0]
    ])

def CZ_matrix(): # Renamed
    return np.array([
        [1,0,0,0],
        [0,1,0,0],
        [0,0,1,0],
        [0,0,0,-1]
    ])

def SWAP_matrix(): # Renamed
    return np.array([
        [1,0,0,0],
        [0,0,1,0],
        [0,1,0,0],
        [0,0,0,1]
    ])

# Helper class for Qubit (optional, but consistent with file 1)
class Qubit:
    def __init__(self, index):
        self.index = index


# =============================== #
#      Quantum Circuit Class      #
# =============================== #

class Circuit:
    def __init__(self, num_qubits):
        self.n = num_qubits
        self.qubits = [Qubit(i) for i in range(num_qubits)] # Added for consistency
        self.reset()

    def reset(self):
        self.state = np.zeros(2**self.n, dtype=np.complex128)
        self.state[0] = 1.0
        self.gates = []

    def add_gate(self, gate):
        # Ensure gate targets are valid for this circuit
        for t in gate.targets:
            if t < 0 or t >= self.n: # Use self.n for consistency
                raise ValueError(f"Qubit index {t} out of range for {self.n} qubits.")
        self.gates.append(gate)

    def run(self):
        for gate in self.gates:
            self.apply_gate(gate)
        # No return needed here, state is updated internally

    def apply_gate(self, gate):
        """Apply a single gate's unitary to the current state vector."""
        if gate.num_targets == 1:
            # One-qubit gate
            target = gate.targets[0]
            n = self.n
            # Calculate indices for pairs (target_qubit=0, target_qubit=1)
            diff = 2 ** (n - 1 - target)
            step = diff * 2
            new_state = self.state.copy()

            # Iterate over pairs of amplitudes where target qubit is 0 vs 1
            # This loop structure correctly handles applying a single-qubit gate
            # across the larger state space.
            for i in range(0, len(self.state), step):
                for j in range(diff):
                    idx0 = i + j             # index where target qubit is 0
                    idx1 = idx0 + diff       # index where target qubit is 1
                    a0, a1 = self.state[idx0], self.state[idx1]

                    # Apply 2x2 matrix U to [a0, a1]
                    new_state[idx0] = gate.matrix[0][0]*a0 + gate.matrix[0][1]*a1
                    new_state[idx1] = gate.matrix[1][0]*a0 + gate.matrix[1][1]*a1
            self.state = new_state

        elif gate.num_targets == 2:
            # Two-qubit gate
            p, q = gate.targets  # the two qubit indices
            n = self.n

            # Determine the positions of the qubits in the state vector indexing
            # Indexing is typically big-endian (most significant bit first).
            # If qubit 0 is the left-most bit, its mask is 2^(n-1), qubit 1 is 2^(n-2), etc.
            mask_p = 2 ** (n - 1 - p)
            mask_q = 2 ** (n - 1 - q)

            new_state = self.state.copy()

            # Iterate through all possible lower bits combinations (excluding p and q)
            # A more efficient way is to iterate directly through the 2x2 sub-blocks
            # but this index-based approach is conceptually clearer for the 4x4 application.
            # We can iterate over all indices and only process those where the bits
            # corresponding to p and q are both 0 in the 'base' index.
            for base in range(len(self.state)):
                # Check if the p-th and q-th bits are both 0 in the current index `base`.
                # This finds the starting index of each 4x4 block in the state vector.
                if (base & mask_p) != 0 or (base & mask_q) != 0:
                    continue # Skip if this is not a 'base' index (where p and q bits are 0)

                # Construct the four indices corresponding to the basis states |p_bit q_bit⟩
                # based on the masks. These must correspond to the order expected by the 4x4 matrix.
                # The standard order is |00>, |01>, |10>, |11>.
                # Assuming the gate matrix is ordered for (qubit p, qubit q):
                # |00> corresponds to index `base`
                # |01> corresponds to index where p bit is 0 and q bit is 1
                # |10> corresponds to index where p bit is 1 and q bit is 0
                # |11> corresponds to index where p bit is 1 and q bit is 1
                # The indices are constructed by adding the masks.
                idx00 = base
                idx01 = base + mask_q
                idx10 = base + mask_p
                idx11 = base + mask_p + mask_q

                # Get current amplitudes for these four basis states
                a00, a01, a10, a11 = self.state[idx00], self.state[idx01], self.state[idx10], self.state[idx11]

                # Apply the 4x4 gate matrix to the vector of these four amplitudes
                result = gate.matrix @ np.array([a00, a01, a10, a11], dtype=np.complex128)

                # Update the new state vector with the transformed amplitudes
                new_state[idx00], new_state[idx01], new_state[idx10], new_state[idx11] = result[0], result[1], result[2], result[3]

            self.state = new_state

        else:
            raise ValueError("Gate with unsupported number of targets.")


    # Removed the incorrect expand_gate method

    def get_statevector(self):
        return self.state.copy() # Return a copy

    def get_probabilities(self):
        return np.abs(self.state)**2

    def measure(self, shots=1024):
        probs = self.get_probabilities()
        basis_states = [format(i, f'0{self.n}b') for i in range(2**self.n)]
        # Ensure probabilities sum to 1 for random.choices
        probs_sum = np.sum(probs)
        if probs_sum > 1e-9: # Avoid division by zero
            normalized_probs = probs / probs_sum
        else:
            normalized_probs = probs # Or handle as needed, e.g., raise error or return empty counts

        samples = random.choices(basis_states, weights=normalized_probs, k=shots)
        return dict(Counter(samples))

    def visualize_probabilities(self, title="State Probabilities"):
        probs = self.get_probabilities()
        basis = [format(i, f'0{self.n}b') for i in range(2**self.n)]
        plt.figure(figsize=(8, 5)) # Added figure size
        plt.bar(basis, probs, color='teal')
        plt.xlabel("Basis States")
        plt.ylabel("Probability")
        plt.title(title)
        plt.grid(axis='y', linestyle='--', alpha=0.7) # Added grid
        plt.show()

    def visualize_state(self):
        """Basic visualization of the current state: Bloch sphere for 1 qubit, or probabilities for multiple qubits."""
        if self.n == 1: # Use self.n for consistency
            # Bloch sphere visualization for single qubit state
            alpha = self.state[0]
            beta = self.state[1] if len(self.state) > 1 else 0 # Should always be len 2 for 1 qubit

            # Ensure the state is normalized for Bloch sphere calculation
            norm = np.sqrt(np.abs(alpha)**2 + np.abs(beta)**2)
            if norm > 1e-9: # Avoid division by zero or very small numbers
                alpha /= norm
                beta /= norm
            else:
                 # Handle the case of a zero state vector (shouldn't happen in unitary evolution)
                 # Or just plot at the origin if the state is somehow zero
                 x, y, z = 0, 0, 0


            # Compute Bloch sphere coordinates (x,y,z) from state alpha | 0> + beta |1>
            x = 2 * np.real(alpha * np.conj(beta))
            y = 2 * np.imag(alpha * np.conj(beta))
            z = np.abs(alpha)**2 - np.abs(beta)**2

            # Plot a 3D Bloch sphere with the state vector
            fig = plt.figure(figsize=(6,6)) # Adjusted figure size
            ax = fig.add_subplot(111, projection='3d')

            # Draw sphere wireframe
            u = np.linspace(0, 2*np.pi, 36)
            v = np.linspace(0, np.pi, 18)
            xs = np.outer(np.cos(u), np.sin(v))
            ys = np.outer(np.sin(u), np.sin(v))
            zs = np.outer(np.ones_like(u), np.cos(v))
            ax.plot_wireframe(xs, ys, zs, color='gray', alpha=0.3)

            # Draw coordinate axes (Scaled slightly for better visualization)
            ax.quiver(0,0,0, 1.2,0,0, color='red', arrow_length_ratio=0.05) # X-axis
            ax.quiver(0,0,0, 0,1.2,0, color='green', arrow_length_ratio=0.05) # Y-axis
            ax.quiver(0,0,0, 0,0,1.2, color='blue', arrow_length_ratio=0.05) # Z-axis
            ax.text(1.3, 0, 0, 'X', color='red', fontsize=12)
            ax.text(0, 1.3, 0, 'Y', color='green', fontsize=12)
            ax.text(0, 0, 1.3, 'Z', color='blue', fontsize=12)

            # Draw state vector
            ax.quiver(0,0,0, x, y, z, color='purple', linewidth=2, arrow_length_ratio=0.1) # Used quiver for vector

            # Set plot limits and labels
            ax.set_xlim([-1, 1])
            ax.set_ylim([-1, 1])
            ax.set_zlim([-1, 1])
            ax.set_xlabel('X')
            ax.set_ylabel('Y')
            ax.set_zlabel('Z')
            ax.set_title('Bloch Sphere')

            # Hide grid lines
            ax.grid(False)
            # Set axis aspect ratio to equal
            ax.set_box_aspect([1,1,1])

            plt.show()

        else:
            # For multiple qubits, visualize probabilities
            self.visualize_probabilities(title="State Probabilities (Multi-Qubit)")


# =============================== #
#            Noise models         #
# =============================== #

def apply_bit_flip(state, p):
    # This noise model application is not integrated into the Circuit class's run method.
    # It would need to be applied *after* a gate operation to the resulting state,
    # or the Circuit class would need a dedicated method to add noise.
    # The current implementation modifies the state vector directly and incorrectly
    # as bit flips should be applied to the basis states probabilistically, not
    # by adding amplitudes.
    # This function is likely not working as intended for simulating noise.
    # A proper noise simulation would involve density matrices or probabilistic
    # application of error operators after each gate.

    # Placeholder/Incorrect implementation
    noisy_state = state.copy()
    for i in range(len(state)):
        if random.random() < p:
            # This is a very simplified and likely incorrect way to simulate bit flip.
            # A bit flip on qubit k means state |...b_k...> flips to |...~b_k...>.
            # The index i represents a basis state. Flipping the LSB of i
            # does not correspond to flipping a specific qubit's bit unless it's qubit 0.
            # A correct implementation would involve applying the error matrix (like X)
            # to the state vector with probability p, and Identity with probability (1-p).
            pass # Do nothing with the incorrect logic

    print("Warning: apply_bit_flip function is a simplified and likely incorrect noise model simulation.")
    return state # Return original state as the implementation is incorrect

def apply_depolarizing(state, p):
     # Similar issues to apply_bit_flip regarding correct noise simulation.
     # A depolarizing channel replaces the state with a mixed state I/d with probability p.
     # Simulating this with state vectors is not standard; it typically requires density matrices.
     d = len(state)
     # This calculation (1-p)*state + p/d * ones is incorrect for depolarizing.
     print("Warning: apply_depolarizing function is a simplified and likely incorrect noise model simulation.")
     return state # Return original state as the implementation is incorrect


# =============================== #
#       Bell state generator      #
# =============================== #

def bell_state(label="Phi+"):
    c = Circuit(2)
    # Use the matrix functions to define the gates
    c.add_gate(OneQubitGate(H(), 0))
    c.add_gate(TwoQubitGate(CNOT_matrix(), 0, 1)) # Use the renamed matrix function
    if label == "Phi+":
        pass
    elif label == "Phi-":
        c.add_gate(OneQubitGate(Z(), 0))
    elif label == "Psi+":
        c.add_gate(OneQubitGate(X(), 1))
    elif label == "Psi-":
        c.add_gate(OneQubitGate(X(), 1))
        c.add_gate(OneQubitGate(Z(), 0))
    else:
        raise ValueError("Unknown Bell state")
    c.run()
    return c

# =============================== #
#         Demonstration           #
# =============================== #

if __name__ == "__main__":

    labels = ["Phi+", "Phi-", "Psi+", "Psi-"]
    shots = 1000

    for label in labels:
        print(f"\n{label} state:")
        c = bell_state(label)
        print("Statevector:", c.get_statevector())
        # Visualize statevector using the new visualize_state method
        c.visualize_state()
        # Measure and print counts
        results = c.measure(shots=shots)
        print(f"Measurement (shots={shots}):", results)
        # You can also visualize the measurement probabilities directly
        c.visualize_probabilities(title=f"{label} state measurement probabilities")

    # Example of a single qubit circuit and visualization
    print("\nSingle qubit example:")
    single_qubit_circuit = Circuit(1)
    single_qubit_circuit.add_gate(OneQubitGate(H(), 0))
    single_qubit_circuit.run()
    print("Statevector after H:", single_qubit_circuit.get_statevector())
    single_qubit_circuit.visualize_state()

    single_qubit_circuit.add_gate(OneQubitGate(Rx(np.pi/2), 0))
    single_qubit_circuit.run()
    print("Statevector after Rx(pi/2):", single_qubit_circuit.get_statevector())
    single_qubit_circuit.visualize_state()



\end{minted}
\end{frame}

\begin{frame}[plain,fragile]
\frametitle{Introduction to \href{{https://qiskit.org/}}{Qiskit}}

For the latest \textbf{qiskit} version see \href{{https://github.com/NuclearTalent/TalentQuantumComputingECT2025/blob/main/doc/pub/Exercises/1_getting_started_with_qiskit.ipynb}}{\nolinkurl{https://github.com/NuclearTalent/TalentQuantumComputingECT2025/blob/main/doc/pub/Exercises/1_getting_started_with_qiskit.ipynb}}
\end{frame}

\begin{frame}[plain,fragile]
\frametitle{Basics of quantum sensing}

\begin{block}{}
In quantum sensing, \textbf{readout functions} refer to the methods used to
extract information about the quantum system after a series of
operations or measurements have been performed.
\end{block} 

\begin{block}{}
The readout process
typically involves measuring the state of quantum bits (qubits) or
other quantum observables, and this information is then used to infer
physical quantities such as time, magnetic fields, temperature, or
other parameters being sensed.
\end{block}
\end{frame}

\begin{frame}[plain,fragile]
\frametitle{Example of Readout Functions in Quantum Sensing}

\begin{block}{}
\textbf{A common example in quantum sensing involves measuring the state of a
qubit after it has interacted with some external field or force.} The
quantum sensor’s response is determined by how the qubit state evolves
due to the applied field, and the readout function is the measurement
procedure that allows us to extract this information.
\end{block}
\end{frame}

\begin{frame}[plain,fragile]
\frametitle{Example 1: Quantum Magnetometry with a Single Qubit}

\begin{block}{}
Let’s consider a typical scenario of quantum magnetometry using a
single qubit that interacts with a magnetic field. We will use the
concept of quantum state measurement as a readout function to infer
the value of the magnetic field.
\end{block}
\end{frame}

\begin{frame}[plain,fragile]
\frametitle{Step 1: Initial Qubit State}

In this example, we use a spin-1/2 system (e.g., a qubit) that is
sensitive to an external magnetic field. A common method of performing
quantum sensing is to apply a quantum gate (e.g., a rotation) to the
qubit and then measure its state.

Assume the initial state of the qubit is
\[
\vert \psi \rangle = \alpha \vert 0\rangle + \beta |1\rangle,
\]

where $\alpha$ and $\beta$ are complex
amplitudes, and $\vert 0\rangle$ and $\vert 1\rangle$
are the computational basis
states.
\end{frame}

\begin{frame}[plain,fragile]
\frametitle{Step 2: Interaction with External Field}

When a magnetic field is applied to the qubit, it will induce a phase
shift in the qubit’s state. The evolution of the qubit’s state can be
modeled using the time evolution operator $U(t)$, which depends on the
interaction with the magnetic field. The evolution can be described
as:
\[
U(t) = \exp{-\imath  \mathcal{H} t}
\]

where $\mathcal{H}$ is the Hamiltonian of the qubit system, which in
the case of a spin-1/2 particle in a magnetic field $B$ could be
\[
\mathcal{H} =-\gamma B \hat{S}_z,
\]
with $\gamma$ being the gyromagnetic ratio and
$\hat{S}_z$ being the spin operator along the $z$-axis.
\end{frame}

\begin{frame}[plain,fragile]
\frametitle{As time evolves}

After time $t$, the qubit’s state becomes:

\[
\vert \psi(t) \rangle = \alpha \vert 0\rangle + \beta \exp{-(i \gamma B t)} \vert 1\rangle,
\]

Here, the phase shift $\exp{-(i \gamma B t)}$ accumulates depending on the applied magnetic field $B$.
\end{frame}

\begin{frame}[plain,fragile]
\frametitle{Step 3: Measurement (Readout Function)}

After the interaction with the magnetic field, the next step is to
measure the qubit. The readout function involves projecting the
quantum state onto a measurement basis, often the computational basis
$\{|0\rangle, |1\rangle\}$. The outcome of the measurement can then be
used to infer information about the external field.
\end{frame}

\begin{frame}[plain,fragile]
\frametitle{Measurement Probability}

\textbf{Measurement Probability}: The probability of measuring the state
$\vert 0\rangle$  or $\vert 1\rangle$ is given by the squared modulus of the
corresponding amplitudes.

\begin{enumerate}
\item Probability of measuring $\vert 0\rangle$: $P(0) = |\alpha|^2$

\item Probability of measuring $\vert 1\rangle$: $P(1) = |\beta \exp{-(i \gamma B t)}|^2 = |\beta|^2$
\end{enumerate}

\noindent
\end{frame}

\begin{frame}[plain,fragile]
\frametitle{Readout Function}

\textbf{Readout Function}: The measurement outcome corresponds to a result that
is used to estimate the magnetic field. For instance, if the qubit
undergoes a rotation due to the magnetic field, the phase shift in the
qubit’s state can be used to infer the magnetic field
strength. Typically, a series of measurements and repeated experiments
are performed to average out the noise and obtain a precise estimate
of $B$.
\end{frame}

\begin{frame}[plain,fragile]
\frametitle{Example 2: Using a Readout Function to Estimate Parameters}

Let’s say the qubit’s state before the measurement is in a
superposition state, and we want to determine the magnetic field
strength by reading out the qubit’s state. After the qubit has
undergone the evolution $U(t)$, we measure it in the computational basis
and repeat the experiment many times to obtain the average outcome.

We could perform parameter estimation by measuring the expected value
of some observable (e.g., $\hat{S}_z)$ and comparing the outcomes to
theoretical predictions.
\end{frame}

\begin{frame}[plain,fragile]
\frametitle{Expectation value of $\hat{S}_z$}

For example, the expectation value of $\hat{S}_z$ (the spin along the $z$-axis)
in the state $\vert \psi(t) \rangle$ is:
\[
\langle \hat{S}_z \rangle = \langle \psi(t) \vert \hat{S}_z \vert \psi(t) \rangle = \frac{1}{2} \left( |\alpha|^2 - |\beta|^2 \right),
\]
\end{frame}

\begin{frame}[plain,fragile]
\frametitle{Phase shift}

Since the magnetic field causes a phase shift $\exp{-(i \gamma B t)}$, the
measurement of the expectation value of $\hat{S}_z$ gives us a way to
infer the magnetic field $B$. By performing a series of measurements and
comparing the observed values to theoretical models, we can estimate
the field strength $B$.
\end{frame}

\begin{frame}[plain,fragile]
\frametitle{One-qubit system}

In the first part of this example, we will analyze our systems using
plain diagonalization and simple analytical manipulations.  Thereafter
we will develop codes and material for performing a quantum computing
simulation of the same systems.

Our first encounter is a simple one-qubit system, described by a simple $2\times 2$ Hamiltonian.

We start with a simple $2\times 2$ Hamiltonian matrix expressed in
terms of Pauli $\bm{X}$, $\bm{Y}$  and $\bm{Z}$ matrices. But before we proceed, a simple reminder is appropriate.
\end{frame}

\begin{frame}[plain,fragile]
\frametitle{Please not again: Single qubit gates}

The Pauli matrices (and gate operations following therefrom) are defined as
\[
	\bm{X} \equiv \sigma_x = \begin{bmatrix}
		0 & 1 \\
		1 & 0
	\end{bmatrix}, \quad
	\bm{Y} \equiv \sigma_y = \begin{bmatrix}
		0 & -i \\
		i & 0
	\end{bmatrix}, \quad
	\bm{Z} \equiv \sigma_z = \begin{bmatrix}
		1 & 0 \\
		0 & -1
	\end{bmatrix}.
\]
\end{frame}

\begin{frame}[plain,fragile]
\frametitle{Pauli gates}

The Pauli-$\bm{X}$ gate is also known as the \textbf{NOT} gate, which flips the state of the qubit.
\begin{align*}
	\bm{X}\vert 0\rangle &= \vert 1\rangle, \\
	\bm{X}\vert 1\rangle &= \vert 0\rangle.	
\end{align*}
The Pauli-$\bm{Y}$ gate flips the bit and multiplies the phase by $ i $. 
\begin{align*}
	\bm{Y}\vert 0\rangle &= i\vert 1\rangle, \\
	\bm{Y}\vert 1\rangle &= -i\vert 0\rangle.
\end{align*}
The Pauli-$\bm{Z}$ gate multiplies only the phase of $\vert 1\rangle$ by $ -1 $.
\begin{align*}
	\bm{Z}\vert 0\rangle &= \vert 0\rangle, \\
	\bm{Z}\vert 1\rangle &= -\vert 1\rangle.
\end{align*}
\end{frame}

\begin{frame}[plain,fragile]
\frametitle{Hadamard gate}

The Hadamard gate is defined as
\[
	\bm{H} = \frac{1}{\sqrt{2}} \begin{bmatrix}
		1 & 1 \\
		1 & -1
	\end{bmatrix}.
\]

It creates a superposition of the $ \vert 0\rangle $ and $ \vert 1\rangle $ states.
\begin{align}
	\bm{H}\vert 0\rangle &= \frac{1}{\sqrt{2}} \left( \vert 0\rangle + \vert 1\rangle \right), \\
	\bm{H}\vert 1\rangle &= \frac{1}{\sqrt{2}} \left( \vert 0\rangle - \vert 1\rangle \right).
\end{align}
Note that we will use $H$ as symbol for the Hadamard gate while we will reserve the notation $\mathcal{H}$ for a given Hamiltonian.
\end{frame}

\begin{frame}[plain,fragile]
\frametitle{Sensing Hamiltonian}

For our discussions, we will assume that the quantum sensor can be
described by the generic Hamiltonian
\[
\mathcal{H}(t) = \mathcal{H}_0 + \mathcal{H}_I(t) + \mathcal{H}_\mathrm{control}(t),
\]

where $\mathcal{H}_0$ is the internal Hamiltonian, $\mathcal{H}_I(t)$ is the
Hamiltonian associated with a signal ($V(t)$ in the notes below), and
$\mathcal{H}_\mathrm{control}(t)$ is the control Hamiltonian.  Following
the above mentioned authors, we will assume that $\mathcal{H}_0$ is known
and that $\mathcal{H}_\mathrm{control}(t)$ can be chosen so as to
manipulate or tune the sensor in a controlled way.

The goal of a quantum sensing experiment is then to infer $V(t)$ from
the effect it has on the actual qubits via its Hamiltonian
$\mathcal{H}_I(t)$, usually by a specific choice of
$\mathcal{H}_\mathrm{control}(t)$.
\end{frame}

\begin{frame}[plain,fragile]
\frametitle{Time-dependent Hamiltonian matrix}

We define a  hermitian  matrix  $\mathcal{H}\in {\mathbb{C}}^{2\times 2}$
\[
\mathcal{H} = \begin{bmatrix} \mathcal{H}_{11} & \mathcal{H}_{12} \\ \mathcal{H}_{21} & \mathcal{H}_{22}
\end{bmatrix}.
\]
We  let $\mathcal{H} = \mathcal{H}_0 + \mathcal{H}_I$, where
\[
\mathcal{H}_0= \begin{bmatrix} E_0 & 0 \\ 0 & E_1\end{bmatrix},
\]
is a diagonal matrix. Similarly,
\[
\mathcal{H}_I(t)= \begin{bmatrix} V_{11}(t) & V_{12}(t) \\ V_{21}(t) & V_{22}(t)\end{bmatrix},
\]
where $V_{ij}(t)$ represent various time-dependent interaction matrix elements and since we have a hermitian matrix, we require that
$V_{21}=V_{12}^*$.
\end{frame}

\begin{frame}[plain,fragile]
\frametitle{Interaction part}

We will now label  the interaction matrix elements, assuming that they have an explicit time dependence.
We define
\begin{align*}
V_{11} & = V_z(t)\\
V_{22} & = -V_z(t)\\
V_{12} & = V_x(t)-\imath V_y(t).
\end{align*}

In the numerical example below we let $V_y(t)=0$, $V_z(t) = tV_z$
and $V_x(t) = tV_x$ with $V_z$ and $V_x$ being real-valued constants to
be determined. In the same numerical example we let $t\in [0,1]$.
\end{frame}

\begin{frame}[plain,fragile]
\frametitle{Non-interacting solution}

We can view $\mathcal{H}_0$ as the non-interacting solution
\[
       \mathcal{H}_0\vert 0 \rangle =E_0\vert 0 \rangle,
\]
and
\[
       \mathcal{H}_0\vert 1\rangle =E_1\vert 1\rangle,
\]
where we have defined the orthogonal computational one-qubit basis states $\vert 0\rangle$ and $\vert 1\rangle$.
\end{frame}

\begin{frame}[plain,fragile]
\frametitle{Rewriting with Pauli matrices}

We rewrite $\mathcal{H}$ (and $\mathcal{H}_0$ and $\mathcal{H}_I$)  via Pauli matrices
\[
\mathcal{H}_0 = \mathcal{E}_{\mathrm{avg}} I -\Delta E \bm{Z}, \quad \mathcal{E}_{\mathrm{avg}} = \frac{E_0
  + E_1}{2}, \; \Delta E = \frac{E_1-E_0}{2},
\]
and
\[
\mathcal{H}_I = V_z(t)\bm{Z} + V_x(t)\bm{X}+V_y(t)\bm{Y},
\]
with $V_z(t) = V_{11}=-V_{22}$, $V_x(t) = \Re (V_{12})$ and $V_y(t) = \Im (V_{12})$.

This is the expression we will discuss in connection with quantum computing simulations. The discussions here, focus mainly on some simpler analytical considerations and simplifications. The numerical solutions are also given by standard eigenvalue solvers.
\end{frame}

\begin{frame}[plain,fragile]
\frametitle{Simple time dependence}

We let our Hamiltonian depend linearly on time  $t$

\[
\mathcal{H}=\mathcal{H}_0+t \mathcal{H}_\mathrm{I},
\]

with $t \in [0,1]$, where the limits $t=0$ and $t=1$
represent the non-interacting (or unperturbed) and fully interacting
system, respectively. This means that the various potential terms are given by $V_i(t)=tV_i$, with $i=\{x,y,z\}$ and $V_i$ are real-valued constants.
\end{frame}

\begin{frame}[plain,fragile]
\frametitle{Exact solution}

Since this a simple $2\times 2$ matrix eigenvalue problem we find the eigenvalues $\lambda_0$ and $\lambda_1$ to be
\[
\lambda_{0,1}=\mathcal{E}_{\mathrm{avg}}\pm \Delta E\sqrt{1+\frac{2V_z(t)}{\Delta E}+\frac{1}{\Delta E^2}(V_z^2(t)+V_x^2(t)+V_y^2(t))}.
\]

If we assume that $\Delta E \gg V_z(t)$ and set $V_x(t)=V_y(t)=0$ for
simplicity and Taylor-expand our square root expression, we obtain
\[
\lambda_{0}=E_0-\frac{1}{2}V_z(t),
\]
\[
\lambda_{1}=E_1+\frac{1}{2}V_z(t),
\]
where we kept only terms up to $\Delta E$. The above problem can however be easily solved numerically, see the code here.
\end{frame}

\begin{frame}[plain,fragile]
\frametitle{Selecting parameters}

The model is an eigenvalue problem with only
two available states.

Here we set the parameters $E_0=0$,
$E_1=4$, $V_{11}=-V_{22}=3$ and $V_{12}=V_{21}=0.2$.

The non-interacting solutions represent our computational basis.
Pertinent to our choice of parameters, is that at $t\geq 2/3$,
the lowest eigenstate is dominated by $\vert 1\rangle$ while the upper
is $\vert 0 \rangle$. At $t=1$ the $\vert 0 \rangle$ mixing of
the lowest eigenvalue is $1\%$ while for $t\leq 2/3$ we have a
$\vert 0 \rangle$ component of more than $90\%$.  The character of the
eigenvectors has therefore been interchanged when passing $z=2/3$. The
value of the parameter $V_{12}$ represents the strength of the coupling
between the two states.  Here we keep only the real part of the non-diagonal term.
\end{frame}

\begin{frame}[plain,fragile]
\frametitle{Setting up the matrix for the simple one-qubit system}

Here we solve the above problem as a standard eigenvalue problem (best seen using the jupyter-notebook)
















































\begin{minted}[fontsize=\fontsize{9pt}{9pt},linenos=false,mathescape,baselinestretch=1.0,fontfamily=tt,xleftmargin=2mm]{python}
import numpy as np
import matplotlib.pyplot as plt
import seaborn as sns; sns.set_theme(font_scale=1.5)
from tqdm import tqdm

sigma_x = np.array([[0, 1], [1, 0]])
sigma_y = np.array([[0, -1j], [1j, 0]])
sigma_z = np.array([[1, 0], [0, -1]])
I = np.eye(2)

def Hamiltonian(lmb):
    E0 = 0
    E1 = 4
    V11 = 3
    V22 = -3
    V12 = 0.2
    V21 = 0.2

    eps = (E0 + E1) / 2
    omega = (E0 - E1) / 2
    c = (V11 + V22) / 2
    omega_z = (V11 - V22) / 2
    omega_x = V12

    H0 = eps * I + omega * sigma_z
    H1 = c * I + omega_z * sigma_z + omega_x * sigma_x
    return H0 + lmb * H1
    
lmbvalues_ana = np.arange(0, 1, 0.01)
eigvals_ana = np.zeros((len(lmbvalues_ana), 2))
for index, lmb in enumerate(lmbvalues_ana):
    H = Hamiltonian(lmb)
    eigen, eigvecs = np.linalg.eig(H)
    permute = eigen.argsort()
    eigvals_ana[index] = eigen[permute]
    eigvecs = eigvecs[:,permute]


fig, axs = plt.subplots(1, 1, figsize=(10, 10))
for i in range(2):
    axs.plot(lmbvalues_ana, eigvals_ana[:,i], label=f'$E_{i}$')
axs.set_xlabel(r'$\lambda$')
axs.set_ylabel('Energy')
axs.legend()
plt.show()


\end{minted}
\end{frame}

\begin{frame}[plain,fragile]
\frametitle{Quantum control  protocol, here tailored to sensing}

\begin{block}{}
\begin{enumerate}
\item The quantum sensor is initialized in some known basis, say $\vert 0\rangle$.

\item The quantum sensor is transformed into the desired initial state $\vert \psi_{\mathrm{Initial}}\rangle$, through an appropriate transformation $\bm{U}_1$. For a single qubit system this could be a Hadamard gate which results in a linear superposition of $\vert 0\rangle$ and $\vert 1\rangle$.

\item The quantum sensor evolves under the Hamiltonian $\mathcal{H}$ for a time $t$. At the end of the sensing period, the sensor is in its final stage (see below) $\vert\psi(t)\rangle=\bm{U}_{\mathcal{H}}(t, 0)\left|\psi_{\mathrm{Initial}}\right\rangle$

\item This quantum state is transformed into a superposition of observable readout states, say a superposition of the one=qubit states $\vert 0\rangle$ and $\vert 1\rangle$, via the action $\bm{U}_2\vert \psi(t)$.

\item The final state of the quantum sensor is read out.

\item Steps 1-5 are repeated and averaged over a large number of cycles $N$ to estimate the final transition probabilities $p$.

\item The transition probability $p$ is measured as a function of time $t$ and used to infer to desired signal $V(t)$.
\end{enumerate}

\noindent
\end{block}
\end{frame}

\begin{frame}[plain,fragile]
\frametitle{Initialization and sensing analysis, Ramsey measurement}

We stay now with the Taylor-approximated solution from the simple
example above (the one-qubit case). We do so in order to illustrate
some of the basic sensing ideas.

To initialize a given system to a known quantum state, we first start
with a known initial state $|0\rangle$. Then, depending on the type of
information that we want to learn about the stimulus, the measurement
scheme to be used, and the physical implementation of the quantum
system, we choose some unitary operator $\bm{U}_{\mathrm{Initial}}$ such that
it transforms our state $|0\rangle$ to a desired initial superposition
state $\left|\psi_{\mathrm{Initial}}\right\rangle=a|0\rangle+b|1\rangle$
for some $a, b \in \mathbb{C}$ such that $|a|^{2}+|b|^{2}=1$.
\end{frame}

\begin{frame}[plain,fragile]
\frametitle{Effects of stimulus}

After the sensing state is initialized, it is exposed to the
environment and evolves according to the time-evolution operator of
the sensing Hamiltonian via the unitary transformation $\bm{U}_{\mathcal{H}}$ as (setting $\hbar=c=e=1$)

\[
|\psi(t)\rangle=\bm{U}_{\mathcal{H}}(t, 0)\left|\psi_{\mathrm{Initial}}\right\rangle.
\]

In general we have
\[ \bm{U}_{\mathcal{H}}=\exp{(\imath\int_{0}^{t}
\mathcal{H}(\tau) d \tau)}.
\]

Here the Hamiltonian could be a
complicated, non-analytical function with a time-dependent $V(t)$
(making $\mathcal{H}$ time-dependent as well).
\end{frame}

\begin{frame}[plain,fragile]
\frametitle{Slowly changing potential}

In the case where $V(t)$ is constant or changes much more slowly than our sensing integration time, we can assume
\[
|\psi(t)\rangle=\bm{U}_{\mathcal{H}}(t, 0)\left|\psi_{\mathrm{Initial}}\right\rangle=\exp{(\imath t \mathcal{H})}\left|\psi_{\mathrm{Initial}}\right\rangle.
\]
The sensing state evolves thus as
\begin{align*}
|\psi(t)\rangle=&\left(\exp{\imath t\left(E_{0}-\frac{1}{2}  V_z\right)}\left|\lambda_{0}\right\rangle\left\langle\lambda_{0}\right|+\exp{\imath t\left(E_{1}+\frac{1}{2}  V_z\right)}\left|\lambda_{1}\right\rangle\left\langle\lambda_{1}\right|\right) \\
& \times \left|\psi_{\mathrm{Initial}}\right\rangle,
\end{align*}
where we have using the spectral decomposition and the final representation of the
sensing Hamiltonian found above.
\end{frame}

\begin{frame}[plain,fragile]
\frametitle{Readout}

After the sensing state has evolved over time in the presence of $V(t)$, it
can be transformed again before a measurement is taken. The first
part, the transformation to some desired read-out state, is performed
by a readout  operator, see discussions in Degen et al., 2017) where
\[
\left|\psi_{\mathrm{Final}}\right\rangle=\bm{U}_{\mathrm{Readout}}|\psi(t)\rangle.
\]

Here the readout operator $\bm{U}_{\mathrm{Readout}}$ is left unspecified.
\end{frame}

\begin{frame}[plain,fragile]
\frametitle{Measurement}

A measurement of this final state $\left|\psi_{\mathrm{Final}}\right\rangle=a^{\prime}|0\rangle+b^{\prime}|1\rangle$ is made with
respect to the basis $\{|0\rangle,|1\rangle\}$ where
$|0\rangle$ is measured with  probability
\[
\left|\left\langle 0 \mid \psi_{\mathrm{Final}}\right\rangle\right|^{2}=\left|a^{\prime}\right|^{2},
\]
and $|1\rangle$ is measured with probability 
\[
\left|\left\langle 1 \mid\psi_{\mathrm{Final}}\right\rangle\right|^{2}=\left|b^{\prime}\right|^{2}.
\]

After this measurement, the sensing state has collapsed into one
of the basis states and  no more information can be gained.
\end{frame}

\begin{frame}[plain,fragile]
\frametitle{Multiple measurements}

However, by having
multiple quantum sensing elements time-evolving together or by
repeating the process many times before the external stimulus $V(t)$
can change, a transition probability
\[
p_{|0\rangle\rightarrow|1\rangle}=\left|\left\langle 1 \mid \psi_{\mathrm{Final}}\right\rangle\right|^{2}=\left|b^{\prime}\right|^{2},
\]
can be estimated. The \emph{sensing} is then achieved by taking a
series of these transition probabilities as a time series, and then
using the results to estimate the sensed stimulus $V(t)$
\end{frame}

\begin{frame}[plain,fragile]
\frametitle{Example}

The simplest mathematical example of quantum sensing is sensing an
external stimulus's effect on the splitting of the energy levels of an
isolated system. Suppose our stimulus is constant and \emph{parallel} with
our sensor, that is we set  $V_z(t)=V_{0}$ and $V_x=0$, and we choose
our initialization and readout preparation operators to be the famous
Hadamard gate

\[
\bm{U}_{H}=\frac{1}{\sqrt{2}}\left[\begin{array}{cc}
1 & 1 \\
1 & -1
\end{array}\right].
\]

Here, the subscript $H$ stands for the Hadamard unitary transformation.
\end{frame}

\begin{frame}[plain,fragile]
\frametitle{Evolution of initial state}

The initial state is 

\[
\left|\psi_{\mathrm{Initial}}\right\rangle=\bm{U}_{H}|0\rangle=\frac{1}{\sqrt{2}}\left[\begin{array}{cc}
1 & 1 \\
1 & -1
\end{array}\right]\begin{bmatrix} 1 \\ 0\end{bmatrix}=\frac{1}{\sqrt{2}}\begin{bmatrix} 1 \\ 1\end{bmatrix}.
\]

This may not necessarily be the same basis into which the system was
initialized, but we  assume it is so and then we only have to keep track
of one basis.
\end{frame}

\begin{frame}[plain,fragile]
\frametitle{State evolution}

The state evolves as
\begin{align*}
|\psi(t)\rangle=&\left(\exp{\imath t\left(E_{0}-\frac{1}{2}  V_z\right)}|0\rangle\langle 0|+\exp{\imath t\left(E_{1}+\frac{1}{2}  V_z\right)}|1\rangle\langle 1|\right)\left|\psi_{\mathrm{Initial}}\right\rangle\\
&=\frac{1}{\sqrt{2}} \exp{\imath t\left(E_{0}-\frac{1}{2}  V_z\right)}\begin{bmatrix}1 \\ \exp{\imath t\left(E_{1}-E_{0}+ V_z\right)}\end{bmatrix}
\end{align*}
\end{frame}

\begin{frame}[plain,fragile]
\frametitle{Preparing for readout}

This is then prepared for readout as

\[
\vert\psi_{\mathrm{Final}}\rangle=\frac{1}{2} \exp{\imath t(E_{0}-\frac{1}{2}  V_z)}
\begin{bmatrix} 1+\exp{\imath t(E_{1}-E_{0}+ V_z)} \\ 1-\exp{(\imath t(E_{1}-E_{0}+ V_z)}\end{bmatrix}.
\]
\end{frame}

\begin{frame}[plain,fragile]
\frametitle{Transition probability}

The transition probability
\begin{align*}
p_{|0\rangle \rightarrow|1\rangle}=\left|\left\langle 1 \mid \psi_{\mathrm{Final}}\right\rangle\right|^{2}=&\left|1-\exp{\imath t\left(E_{1}-E_{0}+ V_z\right)}\right|^{2}\\
&=\frac{1}{2}\left(1-\cos \left(t\left(E_{1}-E_{0}\right)+ V_z\right)\right)
\end{align*}
\end{frame}

\begin{frame}[plain,fragile]
\frametitle{\href{{https://en.wikipedia.org/wiki/Ramsey_interferometry}}{Ramsey interferometry}}

We know the difference in energy between $E_{1}$ and $E_{0}$, either
since we constructed the system or by taking measurements without the
external stimulus $V$, and we can control the time $t$ for which the
system is allowed to evolve under the external stimulus. Then we can
fix $t$ and take many measurements to estimate $p_{|0\rangle
\rightarrow|1\rangle}$, which then makes finding $tV_z$ a simple phase-estimation problem which gives us $
V_z$. The physical implementation of this process is known as Ramsey
Interferometry, and it can be done with arbitary initialization and
readout preparation unitary operators.
\end{frame}

\begin{frame}[plain,fragile]
\frametitle{Benefits of Entanglement}

Up until now, we have said that we take many measurements of
$\left|\psi_{\text {Final }}\right\rangle$ to estimate $p_{|0\rangle
\rightarrow|1\rangle}$, but we have neglected the estimation
process. Assuming we can take $N$ measurements, either by having $N$
experimental apparatuses running in parallel or by taking $N$
different measurements of a (relatively) constant $V$ with a single
apparatus (this is what we will do below), the uncertainty in $p$, denoted as $\sigma_{p}$ (this is a
positive real number; not to be confused with the Pauli matrices),
scales as

\[
\sigma_{p} \propto \frac{1}{\sqrt{N}}
\]
\end{frame}

\begin{frame}[plain,fragile]
\frametitle{Ramsey interferometry}

If we consider Ramsey Interferometry as an example, see \href{{https://en.wikipedia.org/wiki/Ramsey_interferometry}}{\nolinkurl{https://en.wikipedia.org/wiki/Ramsey_interferometry}}, then the
uncertainty in $V_z$, denoted $\sigma_{V}$,
scales as

\[
\sigma_{V} \propto \sigma_{p} \propto \frac{1}{\sqrt{N}}
\]

This relationship is known as the standard quantum limit (SQL)
, see Giovannetti \emph{et al.}, 2011, \href{{https://www.nature.com/articles/nphoton.2011.35}}{\nolinkurl{https://www.nature.com/articles/nphoton.2011.35}},
but can also be explained with the law of
Large Numbers from statistics, where measuring $N$ similarly
distributed, well-behaved random variables gives the sample mean as an
estimator for the population mean and the sample variance divided by
the size of the sample as an uncertainty in the estimate of the
population mean.
\end{frame}

\begin{frame}[plain,fragile]
\frametitle{Key components of code for one qubit}

\begin{block}{Physical System: }
\begin{enumerate}
\item Hamiltonian: $\mathcal{H} = -\frac{\gamma B}{2}Z$ (we will set $\gamma=1$)

\item Initial state: $\vert \psi_{\mathrm{Initial}}\rangle = \frac{1}{\sqrt{2}}(|0\rangle + |1\rangle)$, Hadamard gate
\end{enumerate}

\noindent
\end{block}

\begin{block}{Time Evolution: }
\begin{enumerate}
\item Calculated using matrix exponential: $U_{\mathcal{H}}(t) = \exp{-i\mathcal{H}t}$

\item State at time $t$: $\vert \psi(t)\rangle = U_{\mathcal{H}}(t)\vert\psi_{\mathrm{Initial}} \rangle$
\end{enumerate}

\noindent
\end{block}

\begin{block}{Expectation Values: }
\begin{enumerate}
\item $\langle X\rangle$: Oscillates as $\cos(Bt)$

\item $\langle Y\rangle$: Oscillates as $-\sin(Bt)$

\item $\langle Z\rangle$: Remains constant at $0$
\end{enumerate}

\noindent
\end{block}
\end{frame}

\begin{frame}[plain,fragile]
\frametitle{One-qubit program example (best seen using the jupyter-notebook)}

\begin{minted}[fontsize=\fontsize{9pt}{9pt},linenos=false,mathescape,baselinestretch=1.0,fontfamily=tt,xleftmargin=2mm]{python}
import numpy as np
from scipy.linalg import expm
import matplotlib.pyplot as plt

# Define single-qubit operations, Identity, Pauli, Hadamard and S matrices
Id = np.array([[1, 0], [0, 1]], dtype=complex)
X = np.array([[0, 1], [1, 0]], dtype=complex)
Y = np.array([[0, -1j], [1j, 0]], dtype=complex)
Z = np.array([[1, 0], [0, -1]], dtype=complex)
Had = np.array([[1, 1],[1, -1]], dtype=complex) / np.sqrt(2)
S = np.array([[1, 0],[0, 1j]], dtype=complex)

# Simulation parameters
B = 2 * np.pi  # Magnetic field strength (in angular frequency units)
times = np.linspace(0, 1, 200)  # Time range from 0 to 1 seconds

# Initial state: superposition state via Hadamard gate acting on |0>
psi0 =  np.array([1, 0], dtype=complex) # start with |0> then act with Hadamard
psi0 = Had @ psi0

# Lists to store expectation values
expect_x, expect_y, expect_z = [], [], []

for t in times:
   # Construct Hamiltonian
   H = B * Z/ 2
   # Calculate time evolution operator
   U = expm(-1j * H * t)
   # Evolve the initial state
   psi_t = U @ psi0
   # Calculate expectation values
   expect_x.append(np.vdot(psi_t, X @ psi_t).real)
   expect_y.append(np.vdot(psi_t, Y @ psi_t).real)
   expect_z.append(np.vdot(psi_t, Z @ psi_t).real)
# Plot results
plt.figure(figsize=(10, 6))
plt.plot(times, expect_x, label='<X>')
plt.plot(times, expect_y, label='<Y>')
plt.plot(times, expect_z, label='<Z>')
plt.xlabel('Time (s)')
plt.ylabel('Expectation value')
plt.title('Qubit Spin Evolution in a z-Directional Magnetic Field')
plt.legend()
plt.grid(True)
plt.show()


\end{minted}
\end{frame}

\begin{frame}[plain,fragile]
\frametitle{Interpretation}

\begin{enumerate}
\item The oscillations in $X$ and $Y$ components demonstrate the Larmor precession caused by the magnetic field

\item The frequency of oscillation is directly proportional to the magnetic field strength $B$

\item This forms the basis for quantum sensing: measuring oscillation frequency allows precise determination of $B$
\end{enumerate}

\noindent
\end{frame}

\begin{frame}[plain,fragile]
\frametitle{To use this for sensing (protocol of Degen \emph{et al.,} (2017))}

\begin{enumerate}
\item Prepare the qubit in a known superposition state

\item Let it evolve in the magnetic field for a known time

\item Measure the expectation values

\item Determine (here) $B$ from the oscillation frequency
\end{enumerate}

\noindent
\end{frame}

\begin{frame}[plain,fragile]
\frametitle{Feel free to play around with this code}

\begin{enumerate}
\item Change $B$ value to see different oscillation frequencies

\item Adjust \emph{time} array to observe different numbers of oscillations

\item Add noise to simulate real-world sensing scenarios

\item Implement actual measurement simulations instead of expectation values
\end{enumerate}

\noindent
This code provides a fundamental demonstration of quantum sensing
principles using a simple qubit system. Real-world implementations
would typically use more sophisticated techniques like Ramsey
interferometry or dynamical decoupling for enhanced sensitivity.
\end{frame}

\begin{frame}[plain,fragile]
\frametitle{More than one qubit}

The nature of quantum systems allows for more information to be
extracted by exploiting entanglement between quantum systems. This is
the fundamental basis for the benefits of quantum computing over
classical computing, and quantum sensing has similar benefits over
classical sensing. Suppose we return to the example above, but rather
than initializing $N$ sensing qubits separately, we initialize
$\frac{N}{n}$ groups each with $n$ entangled quantum systems. Then we
have

\[
\left|\psi_{\text {Init }}\right\rangle=\frac{1}{\sqrt{2^{n}}}\left(|0\rangle^{\otimes n}+|1\rangle^{\otimes n}\right),
\]
where $|0\rangle^{\otimes n}=|0\rangle \otimes \ldots \otimes|0\rangle, n$ times.
\end{frame}

\begin{frame}[plain,fragile]
\frametitle{After initialization}

After initialization, each of the $n$ sensing qubits evolves to pick up a relative phase factor of $\exp{\imath t\left(E_{1}-E_{0}+ V_z\right)}$, which combined results in
\[
|\psi(t)\rangle=\mathcal{N}\left(|0\rangle^{\otimes n}+\exp{n \imath t\left(E_{1}-E_{0}+ V_z\right)}|1\rangle^{\otimes n}\right)
\]

where $\mathcal{N}$ is just a factor to take care of normalization.
\end{frame}

\begin{frame}[plain,fragile]
\frametitle{Transition probability}

The transition probability
\[
p_{|0\rangle \rightarrow|1\rangle}=\left|\left\langle 1 \mid \psi_{\text {Final }}\right\rangle\right|^{2}=\frac{1}{2}\left(1-\cos \left(t n\left(E_{1}-E_{0}\right)+n  V_z\right)\right)
\]
\end{frame}

\begin{frame}[plain,fragile]
\frametitle{Role of entanglement}

From this, we can see that through entangling $n$ sensing qubits, the
\textbf{signal} we are trying to sense increases from $V_z \rightarrow n
V_z$, and with $\frac{N}{n}$ total measurements,
\[
\sigma_{V} \propto \frac{1}{n} \sigma_{p} \propto \frac{1}{n}\left(\frac{1}{\sqrt{\frac{N}{n}}}\right)=\frac{1}{\sqrt{N n}}
\]
which means the error decreased by a factor of $\sqrt{n}$. In the case where $n=N$, the uncertainty now scales as
\[
\sigma_{V} \propto \frac{1}{N}
\]
which is known as the Heisenberg limit, and is the
quantum-mechanically limited, maximal amount of information one can
get from taking $n$ quantum sensing measurements, see again Giovannetti \emph{et al.}, at \href{{https://www.nature.com/articles/nphoton.2011.35}}{\nolinkurl{https://www.nature.com/articles/nphoton.2011.35}}.
\end{frame}

\begin{frame}[plain,fragile]
\frametitle{Two-qubit program example (best seen using the jupyter-notebook)}

\begin{minted}[fontsize=\fontsize{9pt}{9pt},linenos=false,mathescape,baselinestretch=1.0,fontfamily=tt,xleftmargin=2mm]{python}
import numpy as np
from scipy.linalg import expm
import matplotlib.pyplot as plt

# Define single-qubit operations, Identity, Pauli, Hadamard and S matrices
Id = np.array([[1, 0], [0, 1]], dtype=complex)
X = np.array([[0, 1], [1, 0]], dtype=complex)
Y = np.array([[0, -1j], [1j, 0]], dtype=complex)
Z = np.array([[1, 0], [0, -1]], dtype=complex)
Had = np.array([[1, 1],[1, -1]], dtype=complex) / np.sqrt(2)
S = np.array([[1, 0],[0, 1j]], dtype=complex)
# Define two-qubit gates
CNOT01 = np.array([[1, 0, 0, 0], [0, 1, 0, 0], [0, 0, 0, 1], [0, 0, 1, 0]], dtype=complex)
CNOT10 = np.array([[1, 0, 0, 0], [0, 0, 0, 1], [0, 0, 1, 0], [0, 1, 0, 0]], dtype=complex)
SWAP = np.array([[1, 0, 0, 0], [0, 0, 1, 0], [0, 1, 0, 0], [0, 0, 0, 1]], dtype=complex)

times = np.linspace(0, 1, 100)  # Time range from 0 to 1 seconds

# Initial state for each qubit
psi_1 =  np.array([1, 0], dtype=complex) # start with |0> for qubits 1 and 2
psi_2 =  np.array([1, 0], dtype=complex) # start with |0> for qubits 1 and 2
# possible basis states for measurements
basis_00 = np.array([1, 0, 0, 0], dtype=complex)
basis_01 = np.array([0, 1, 0, 0], dtype=complex)
basis_10 = np.array([0, 0, 1, 0], dtype=complex)
basis_11 = np.array([0, 0, 0, 1], dtype=complex)
# then act with Hadamard on first qubit only
psi_1 = Had @ psi_1
# Initial two-qubit state
Psi_0 = np.kron(psi_1,psi_2)
# Then we act on this state in order to get a Bell state 1/sqrt(2)(|00>+|11)
Psi_0 = CNOT01 @ Psi_0
# Define parameters of Hamiltonian (Time-independent)
B = 1.0  # Strength of the magnetic field (in arbitrary units)
omega = B  # Frequency associated with the magnetic field
# Constructing the Hamiltonian H = -omega/2 * (Z * I + I * Z)
H_z_I = -omega / 2 * np.kron(Z, Id)  # Z * I
I_H_z = -omega / 2 * np.kron(Id, Z)    # I * Z
# Total Hamiltonian, try a more complicated one, this one is very simple
H = H_z_I + I_H_z   
# Lists to store expectation values
expect_00, expect_11 = [], []
for t in times:
   # Calculate time evolution operator
   U = expm(-1j * H * t)
   # Evolve the initial state (if you have a time-dependent H, remember to integrate)
   Psi_t = U @ Psi_0
   # Calculate probabilities of measuring specific states: P(|00>) and P(|11>)
   # Calculate expectation values (boring case here)
   expect_00.append(abs(np.dot(basis_00.conj(), Psi_t))**2)
   expect_11.append(abs(np.dot(basis_11.conj(), Psi_t))**2)

# Plotting results using matplotlib
plt.figure(figsize=(10, 6))
plt.plot(times, expect_00, label='Probability |00>')
plt.plot(times, expect_11, label='Probability |11>')
plt.xlabel('Time')
plt.ylabel('Probability')
plt.title('Quantum Sensing Simulation')
plt.legend()
plt.show()


\end{minted}
\end{frame}

\begin{frame}[plain,fragile]
\frametitle{Analysis and additions}

The result here is pretty boring, we just get $0.5$, as expected. The field acts only in the $z$-direction and is a constant. The probabilities are thus left unchanged.
Feel free to change the Hamiltonian. Here's a list of suggestion:
\begin{enumerate}
\item More realistic Hamiltonian (keep in mind that if you make it time-dependent, you need to integrate over time. Different approximation exist.

\item Change the measurements to say making a measurement on only one of the qubits

\item More realistic readout functions and series of measurements as experiments

\item Study different entangled states

\item Find the field which acts on the system! How wuld you do that?

\item Study the Fisher entropy and much more
\end{enumerate}

\noindent
\end{frame}

\begin{frame}[plain,fragile]
\frametitle{Conclusion}

\begin{block}{}
The readout functions in quantum sensing serve as the crucial step in
extracting classical information from a quantum system after a set of
operations. In quantum magnetometry, for instance, the readout is
typically performed by measuring the qubit in the computational basis
and using the outcome (such as the probabilities of measuring
$\vert 0\rangle$ or $\vert 1\rangle$) to infer properties like the magnetic field
strength. In more complex systems, the readout functions could involve
more sophisticated measurements such as those based on quantum
tomography, where the state of the system is fully reconstructed from
measurement data.
\end{block}
\end{frame}

\begin{frame}[plain,fragile]
\frametitle{Other types of sensors}

This process can be generalized to other types of quantum sensors,
such as those measuring electric fields, temperature, or time. The
general concept remains: we manipulate a quantum system, measure it,
and extract classical information to sense the desired physical
quantity.

The examples below illustrate the diversity of quantum sensing
applications, where quantum states (entanglement, superposition,
squeezing) or quantum systems (atoms, spins, photons) enable
breakthroughs in precision beyond classical limits.
\end{frame}

\begin{frame}[plain,fragile]
\frametitle{Magnetic Field Sensing with Nitrogen-Vacancy (NV) Centers}

\begin{itemize}
\item Parameter: Magnetic field strength.  
\end{itemize}

\noindent
\begin{block}{}
Method/platform: NV centers in diamond have electron spins sensitive to magnetic fields.
By optically detecting spin state changes (via fluorescence), magnetic
fields at the nanoscale are estimated with high spatial resolution,
useful in material science and biomedical imaging.
\end{block}
\end{frame}

\begin{frame}[plain,fragile]
\frametitle{Atomic Clocks}

\begin{itemize}
\item Parameter: Time/frequency.  
\end{itemize}

\noindent
\begin{block}{}
Method/platform: Atomic transitions (e.g., in cesium or rubidium) serve as frequency standards. Quantum superposition states are probed to lock oscillator frequencies, enabling ultra-precise timekeeping critical for GPS and telecommunications.
\end{block}
\end{frame}

\begin{frame}[plain,fragile]
\frametitle{Gravitational Wave Detection (LIGO)}

\begin{itemize}
\item Parameter: Phase shift induced by spacetime ripples.  
\end{itemize}

\noindent
\begin{block}{}
Method/platform: Squeezed light reduces quantum noise in interferometers, enhancing sensitivity to minute phase shifts caused by gravitational waves, pushing measurements below the standard quantum limit.
\end{block}
\end{frame}

\begin{frame}[plain,fragile]
\frametitle{Quantum Thermometry}

\begin{itemize}
\item Parameter: Temperature.  
\end{itemize}

\noindent
\begin{block}{}
Method/platform**: Quantum probes like trapped ions or superconducting qubits exploit temperature-dependent decoherence or energy-level shifts to measure microkelvin-scale temperatures in cryogenic systems.
\end{block}
\end{frame}

\begin{frame}[plain,fragile]
\frametitle{Entangled Photon Interferometry}

\begin{itemize}
\item Parameter: Optical phase shifts.
\end{itemize}

\noindent
\begin{block}{}
Method/platform: Entangled photons in interferometers achieve sub-shot-noise precision, enabling enhanced measurements of distances or refractive indices for applications in metrology and imaging.
\end{block}
\end{frame}

\begin{frame}[plain,fragile]
\frametitle{Electric Field Sensing with Rydberg Atoms}

\begin{itemize}
\item Parameter: Electric field strength.  
\end{itemize}

\noindent
\begin{block}{}
Method/platform: Rydberg atoms, highly sensitive to electric fields due to their large polarizability, detect field-induced Stark shifts via microwave spectroscopy, useful in electrometry and communications.
\end{block}
\end{frame}

\begin{frame}[plain,fragile]
\frametitle{Quantum Gyroscopes}

\begin{itemize}
\item Parameter: Rotation rate.  
\end{itemize}

\noindent
\begin{block}{}
Method/platform: Cold atom interferometers or entangled particles exploit the Sagnac effect to measure rotation with quantum-enhanced precision, advancing inertial navigation systems.
\end{block}
\end{frame}

\begin{frame}[plain,fragile]
\frametitle{Molecular Concentration Sensing}

\begin{itemize}
\item Parameter: Chemical concentration.  
\end{itemize}

\noindent
\begin{block}{}
Method/platform: Spin defects in diamond (e.g., NV centers) detect local magnetic perturbations from target molecules, enabling nanoscale NMR spectroscopy for biological or chemical analysis.
\end{block}
\end{frame}

\begin{frame}[plain,fragile]
\frametitle{Superconducting Quantum Interference Devices (SQUIDs)}

\begin{itemize}
\item Parameter: Magnetic flux.  
\end{itemize}

\noindent
\begin{block}{}
Method/platform: SQUIDs leverage flux quantization and Josephson junctions to measure extremely weak magnetic fields, applied in magnetoencephalography (MEG) for brain activity mapping.
\end{block}
\end{frame}

\begin{frame}[plain,fragile]
\frametitle{A more refined example}

\begin{minted}[fontsize=\fontsize{9pt}{9pt},linenos=false,mathescape,baselinestretch=1.0,fontfamily=tt,xleftmargin=2mm]{python}
import numpy as np
import matplotlib.pyplot as plt
from scipy.linalg import expm

class QuantumSensor:
    def __init__(self, shots=1000, time_steps=100):
        self.shots = shots
        self.time_steps = time_steps
        self.state = self.create_bell_state()
        self.measurement_results = np.zeros(shots)
        self.time_points = np.linspace(0, 10, self.time_steps)
        # B is a 3xN array where N is time_steps. B[:, t] gives the magnetic field vector at time t.
        self.B = np.array([np.cos(self.time_points), np.sin(self.time_points), np.zeros_like(self.time_points)])

    @staticmethod
    def create_bell_state():
        """Create an entangled Bell state (|00> + |11>)/sqrt(2)."""
        # Representing states in the computational basis |00>, |01>, |10>, |11>
        return (1/np.sqrt(2)) * (np.array([[1], [0], [0], [0]]) +
                                     np.array([[0], [0], [0], [1]]))

    @staticmethod
    def hamiltonian(B):
        """Create the Hamiltonian for the current magnetic field for a two-qubit system."""
        hbar = 1  # Planck's constant
        X = np.array([[0, 1], [1, 0]], dtype=complex)
        Y = np.array([[0, -1j], [1j, 0]], dtype=complex)
        Z = np.array([[1, 0], [0, -1]], dtype=complex)
        I = np.eye(2, dtype=complex) # 2x2 Identity matrix

        # Hamiltonian for two independent qubits interacting with the magnetic field
        # H = H1 + H2, where H1 = -hbar * B . sigma_1 and H2 = -hbar * B . sigma_2
        # In the two-qubit space (4x4 matrix):
        # H = -hbar * (Bx * (sigma_x tensor I) + By * (sigma_y tensor I) + Bz * (sigma_z tensor I) +
        #           Bx * (I tensor sigma_x) + By * (I tensor sigma_y) + Bz * (I tensor sigma_z))
        H1 = -hbar * (B[0] * np.kron(X, I) + B[1] * np.kron(Y, I) + B[2] * np.kron(Z, I))
        H2 = -hbar * (B[0] * np.kron(I, X) + B[1] * np.kron(I, Y) + B[2] * np.kron(I, Z))

        return H1 + H2

    @staticmethod
    def evolve_state(state, H, time):
        """Evolve the state using the unitary operator."""
        U = expm(-1j * H * time)
        return U @ state

    def measure(self, state):
        """Measure the state in the standard basis {|00>, |01>, |10>, |11>}."""
        # Ensure state is a column vector for probability calculation
        state_vector = state.flatten()
        probabilities = np.abs(state_vector)**2
        # Ensure probabilities sum to 1 (due to potential floating point inaccuracies)
        probabilities /= np.sum(probabilities)
        # Choose outcome based on probabilities. outcomes correspond to indices 0, 1, 2, 3
        outcome = np.random.choice(range(len(probabilities)), p=probabilities)
        # Return the outcome (index) and the corresponding state vector component
        return outcome, state_vector[outcome]


    def simulate(self):
        """Run the simulation for a given number of shots."""
        for shot in range(self.shots):
            evolved_state = self.state # Start each shot with the initial Bell state
            # The Hamiltonian is time-dependent, so we need to apply the evolution operator
            # for each small time step. The total evolution U(T) = U(dt_N) * ... * U(dt_1).
            # We are assuming a small constant time step dt = 0.1.
            for t in range(self.time_steps):
                # Get the magnetic field vector at the current time point
                current_B = self.B[:, t]
                # Calculate the Hamiltonian for the current magnetic field
                H = self.hamiltonian(current_B)
                # Evolve the state for this small time step
                evolved_state = self.evolve_state(evolved_state, H, 0.1)

            # Measure the final state after the total evolution time
            outcome, _ = self.measure(evolved_state)
            self.measurement_results[shot] = outcome

    def plot_results(self):
        """Plot the histogram of measurement outcomes."""
        plt.figure(figsize=(10, 6))
        # The possible outcomes for a two-qubit measurement are 0, 1, 2, 3
        plt.hist(self.measurement_results, bins=np.arange(5) - 0.5, density=True, rwidth=0.8)
        plt.xticks(range(4))
        plt.xlabel('Measurement Outcome (|00>, |01>, |10>, |11>)')
        plt.ylabel('Probability')
        plt.title(f'Histogram of Measurement Outcomes over {self.shots} Shots')
        plt.grid(axis='y', alpha=0.75)
        plt.show()

# Main execution
if __name__ == "__main__":
    try:
        from scipy.linalg import expm
    except ImportError:
        print("Please install scipy: !pip install scipy")
        exit()

    sensor = QuantumSensor(shots=1000, time_steps=100)
    sensor.simulate()
    sensor.plot_results()



\end{minted}
\end{frame}

\end{document}
